\documentclass[]{article}
\usepackage{lmodern}
\usepackage{amssymb,amsmath}
\usepackage{ifxetex,ifluatex}
\usepackage{fixltx2e} % provides \textsubscript
\ifnum 0\ifxetex 1\fi\ifluatex 1\fi=0 % if pdftex
  \usepackage[T1]{fontenc}
  \usepackage[utf8]{inputenc}
\else % if luatex or xelatex
  \ifxetex
    \usepackage{mathspec}
  \else
    \usepackage{fontspec}
  \fi
  \defaultfontfeatures{Ligatures=TeX,Scale=MatchLowercase}
\fi
% use upquote if available, for straight quotes in verbatim environments
\IfFileExists{upquote.sty}{\usepackage{upquote}}{}
% use microtype if available
\IfFileExists{microtype.sty}{%
\usepackage{microtype}
\UseMicrotypeSet[protrusion]{basicmath} % disable protrusion for tt fonts
}{}
\usepackage[margin=1in]{geometry}
\usepackage{hyperref}
\hypersetup{unicode=true,
            pdfborder={0 0 0},
            breaklinks=true}
\urlstyle{same}  % don't use monospace font for urls
\usepackage{graphicx,grffile}
\makeatletter
\def\maxwidth{\ifdim\Gin@nat@width>\linewidth\linewidth\else\Gin@nat@width\fi}
\def\maxheight{\ifdim\Gin@nat@height>\textheight\textheight\else\Gin@nat@height\fi}
\makeatother
% Scale images if necessary, so that they will not overflow the page
% margins by default, and it is still possible to overwrite the defaults
% using explicit options in \includegraphics[width, height, ...]{}
\setkeys{Gin}{width=\maxwidth,height=\maxheight,keepaspectratio}
\IfFileExists{parskip.sty}{%
\usepackage{parskip}
}{% else
\setlength{\parindent}{0pt}
\setlength{\parskip}{6pt plus 2pt minus 1pt}
}
\setlength{\emergencystretch}{3em}  % prevent overfull lines
\providecommand{\tightlist}{%
  \setlength{\itemsep}{0pt}\setlength{\parskip}{0pt}}
\setcounter{secnumdepth}{0}
% Redefines (sub)paragraphs to behave more like sections
\ifx\paragraph\undefined\else
\let\oldparagraph\paragraph
\renewcommand{\paragraph}[1]{\oldparagraph{#1}\mbox{}}
\fi
\ifx\subparagraph\undefined\else
\let\oldsubparagraph\subparagraph
\renewcommand{\subparagraph}[1]{\oldsubparagraph{#1}\mbox{}}
\fi

%%% Use protect on footnotes to avoid problems with footnotes in titles
\let\rmarkdownfootnote\footnote%
\def\footnote{\protect\rmarkdownfootnote}

%%% Change title format to be more compact
\usepackage{titling}

% Create subtitle command for use in maketitle
\newcommand{\subtitle}[1]{
  \posttitle{
    \begin{center}\large#1\end{center}
    }
}

\setlength{\droptitle}{-2em}
  \title{}
  \pretitle{\vspace{\droptitle}}
  \posttitle{}
  \author{}
  \preauthor{}\postauthor{}
  \date{}
  \predate{}\postdate{}


\begin{document}

\section{DATA PROCESSING FOR MASTER'S
THESIS}\label{data-processing-for-masters-thesis}

\subsubsection{data from 2 experiments (two-choice maze) - experiment 1
had 24 subjects and experiment 2 had 12
subjects}\label{data-from-2-experiments-two-choice-maze---experiment-1-had-24-subjects-and-experiment-2-had-12-subjects}

\subsubsection{subjects were trained daily 3 trials per
day}\label{subjects-were-trained-daily-3-trials-per-day}

\subsubsection{data collected includes success (0 or 1 for each trial, 0
to 3 for each day), non-contingent errors (0 or 1 for each trial, 0 to 3
for each day), and position errors (no restricting
parameters)}\label{data-collected-includes-success-0-or-1-for-each-trial-0-to-3-for-each-day-non-contingent-errors-0-or-1-for-each-trial-0-to-3-for-each-day-and-position-errors-no-restricting-parameters}

\subsection{random effects include
SUB}\label{random-effects-include-sub}

\subsection{fixed effects include SEX and
DAY}\label{fixed-effects-include-sex-and-day}

\subsubsection{first upload data for each
experiment}\label{first-upload-data-for-each-experiment}

exp1 \textless{}- read.csv(``exp1.csv'', header = TRUE) exp2
\textless{}- read.csv(``exp2.csv'', header = TRUE)

\subsubsection{the model}\label{the-model}

\section{the goal is to model success as a function of
day}\label{the-goal-is-to-model-success-as-a-function-of-day}

glm \textless{}- glmer(SUCCESS \textasciitilde{} DAY + SEX +
(1\textbar{}SUB), data = exp1))


\end{document}
