\documentclass[]{article}
\usepackage{lmodern}
\usepackage{amssymb,amsmath}
\usepackage{ifxetex,ifluatex}
\usepackage{fixltx2e} % provides \textsubscript
\ifnum 0\ifxetex 1\fi\ifluatex 1\fi=0 % if pdftex
  \usepackage[T1]{fontenc}
  \usepackage[utf8]{inputenc}
\else % if luatex or xelatex
  \ifxetex
    \usepackage{mathspec}
  \else
    \usepackage{fontspec}
  \fi
  \defaultfontfeatures{Ligatures=TeX,Scale=MatchLowercase}
\fi
% use upquote if available, for straight quotes in verbatim environments
\IfFileExists{upquote.sty}{\usepackage{upquote}}{}
% use microtype if available
\IfFileExists{microtype.sty}{%
\usepackage{microtype}
\UseMicrotypeSet[protrusion]{basicmath} % disable protrusion for tt fonts
}{}
\usepackage[margin=1in]{geometry}
\usepackage{hyperref}
\hypersetup{unicode=true,
            pdfborder={0 0 0},
            breaklinks=true}
\urlstyle{same}  % don't use monospace font for urls
\usepackage{color}
\usepackage{fancyvrb}
\newcommand{\VerbBar}{|}
\newcommand{\VERB}{\Verb[commandchars=\\\{\}]}
\DefineVerbatimEnvironment{Highlighting}{Verbatim}{commandchars=\\\{\}}
% Add ',fontsize=\small' for more characters per line
\usepackage{framed}
\definecolor{shadecolor}{RGB}{248,248,248}
\newenvironment{Shaded}{\begin{snugshade}}{\end{snugshade}}
\newcommand{\KeywordTok}[1]{\textcolor[rgb]{0.13,0.29,0.53}{\textbf{#1}}}
\newcommand{\DataTypeTok}[1]{\textcolor[rgb]{0.13,0.29,0.53}{#1}}
\newcommand{\DecValTok}[1]{\textcolor[rgb]{0.00,0.00,0.81}{#1}}
\newcommand{\BaseNTok}[1]{\textcolor[rgb]{0.00,0.00,0.81}{#1}}
\newcommand{\FloatTok}[1]{\textcolor[rgb]{0.00,0.00,0.81}{#1}}
\newcommand{\ConstantTok}[1]{\textcolor[rgb]{0.00,0.00,0.00}{#1}}
\newcommand{\CharTok}[1]{\textcolor[rgb]{0.31,0.60,0.02}{#1}}
\newcommand{\SpecialCharTok}[1]{\textcolor[rgb]{0.00,0.00,0.00}{#1}}
\newcommand{\StringTok}[1]{\textcolor[rgb]{0.31,0.60,0.02}{#1}}
\newcommand{\VerbatimStringTok}[1]{\textcolor[rgb]{0.31,0.60,0.02}{#1}}
\newcommand{\SpecialStringTok}[1]{\textcolor[rgb]{0.31,0.60,0.02}{#1}}
\newcommand{\ImportTok}[1]{#1}
\newcommand{\CommentTok}[1]{\textcolor[rgb]{0.56,0.35,0.01}{\textit{#1}}}
\newcommand{\DocumentationTok}[1]{\textcolor[rgb]{0.56,0.35,0.01}{\textbf{\textit{#1}}}}
\newcommand{\AnnotationTok}[1]{\textcolor[rgb]{0.56,0.35,0.01}{\textbf{\textit{#1}}}}
\newcommand{\CommentVarTok}[1]{\textcolor[rgb]{0.56,0.35,0.01}{\textbf{\textit{#1}}}}
\newcommand{\OtherTok}[1]{\textcolor[rgb]{0.56,0.35,0.01}{#1}}
\newcommand{\FunctionTok}[1]{\textcolor[rgb]{0.00,0.00,0.00}{#1}}
\newcommand{\VariableTok}[1]{\textcolor[rgb]{0.00,0.00,0.00}{#1}}
\newcommand{\ControlFlowTok}[1]{\textcolor[rgb]{0.13,0.29,0.53}{\textbf{#1}}}
\newcommand{\OperatorTok}[1]{\textcolor[rgb]{0.81,0.36,0.00}{\textbf{#1}}}
\newcommand{\BuiltInTok}[1]{#1}
\newcommand{\ExtensionTok}[1]{#1}
\newcommand{\PreprocessorTok}[1]{\textcolor[rgb]{0.56,0.35,0.01}{\textit{#1}}}
\newcommand{\AttributeTok}[1]{\textcolor[rgb]{0.77,0.63,0.00}{#1}}
\newcommand{\RegionMarkerTok}[1]{#1}
\newcommand{\InformationTok}[1]{\textcolor[rgb]{0.56,0.35,0.01}{\textbf{\textit{#1}}}}
\newcommand{\WarningTok}[1]{\textcolor[rgb]{0.56,0.35,0.01}{\textbf{\textit{#1}}}}
\newcommand{\AlertTok}[1]{\textcolor[rgb]{0.94,0.16,0.16}{#1}}
\newcommand{\ErrorTok}[1]{\textcolor[rgb]{0.64,0.00,0.00}{\textbf{#1}}}
\newcommand{\NormalTok}[1]{#1}
\usepackage{graphicx,grffile}
\makeatletter
\def\maxwidth{\ifdim\Gin@nat@width>\linewidth\linewidth\else\Gin@nat@width\fi}
\def\maxheight{\ifdim\Gin@nat@height>\textheight\textheight\else\Gin@nat@height\fi}
\makeatother
% Scale images if necessary, so that they will not overflow the page
% margins by default, and it is still possible to overwrite the defaults
% using explicit options in \includegraphics[width, height, ...]{}
\setkeys{Gin}{width=\maxwidth,height=\maxheight,keepaspectratio}
\IfFileExists{parskip.sty}{%
\usepackage{parskip}
}{% else
\setlength{\parindent}{0pt}
\setlength{\parskip}{6pt plus 2pt minus 1pt}
}
\setlength{\emergencystretch}{3em}  % prevent overfull lines
\providecommand{\tightlist}{%
  \setlength{\itemsep}{0pt}\setlength{\parskip}{0pt}}
\setcounter{secnumdepth}{0}
% Redefines (sub)paragraphs to behave more like sections
\ifx\paragraph\undefined\else
\let\oldparagraph\paragraph
\renewcommand{\paragraph}[1]{\oldparagraph{#1}\mbox{}}
\fi
\ifx\subparagraph\undefined\else
\let\oldsubparagraph\subparagraph
\renewcommand{\subparagraph}[1]{\oldsubparagraph{#1}\mbox{}}
\fi

%%% Use protect on footnotes to avoid problems with footnotes in titles
\let\rmarkdownfootnote\footnote%
\def\footnote{\protect\rmarkdownfootnote}

%%% Change title format to be more compact
\usepackage{titling}

% Create subtitle command for use in maketitle
\newcommand{\subtitle}[1]{
  \posttitle{
    \begin{center}\large#1\end{center}
    }
}

\setlength{\droptitle}{-2em}
  \title{}
  \pretitle{\vspace{\droptitle}}
  \posttitle{}
  \author{}
  \preauthor{}\postauthor{}
  \date{}
  \predate{}\postdate{}


\begin{document}

\section{LEARNING IN TUNGARA FROGS: CAN THEY LEARN A
MAZE?}\label{learning-in-tungara-frogs-can-they-learn-a-maze}

\subsection{initial hypothesis: females are better learners than males
based on behavioral requirements for
breeding.}\label{initial-hypothesis-females-are-better-learners-than-males-based-on-behavioral-requirements-for-breeding.}

\subsubsection{data from 2 experiments (two-choice mazes) - experiment 1
had 24 subjects and experiment 2 had 12
subjects}\label{data-from-2-experiments-two-choice-mazes---experiment-1-had-24-subjects-and-experiment-2-had-12-subjects}

\subsubsection{subjects were trained daily 3 trials per
day}\label{subjects-were-trained-daily-3-trials-per-day}

\subsubsection{data collected includes success (0 or 1 for each trial, 0
to 3 for each day), non-contingent errors (0 or 1 for each trial, 0 to 3
for each day), and position errors (no restricting
parameters)}\label{data-collected-includes-success-0-or-1-for-each-trial-0-to-3-for-each-day-non-contingent-errors-0-or-1-for-each-trial-0-to-3-for-each-day-and-position-errors-no-restricting-parameters}

\section{main relationships of
interest}\label{main-relationships-of-interest}

\subsubsection{how is success predicted by which day of training it
occured
on?}\label{how-is-success-predicted-by-which-day-of-training-it-occured-on}

\subsubsection{how is success predicted by
sex?}\label{how-is-success-predicted-by-sex}

\subsubsection{how are errors predicted by
day/sex?}\label{how-are-errors-predicted-by-daysex}

\subsubsection{how do errors vary with
success?}\label{how-do-errors-vary-with-success}

\subsection{dependent/response variables are SUCCESS, P\_errors and
NC\_errors}\label{dependentresponse-variables-are-success-p_errors-and-nc_errors}

\subsection{random effect is SUB}\label{random-effect-is-sub}

\subsection{fixed effect is SEX}\label{fixed-effect-is-sex}

\section{first, I uploaded the data for each of the
experiments}\label{first-i-uploaded-the-data-for-each-of-the-experiments}

I've decided to use centered and scale time variables as it usually
provides more stable fits with GLMM - thanksfor this tip James!

\begin{Shaded}
\begin{Highlighting}[]
\KeywordTok{library}\NormalTok{(dplyr)}
\end{Highlighting}
\end{Shaded}

\begin{verbatim}
## 
## Attaching package: 'dplyr'
\end{verbatim}

\begin{verbatim}
## The following objects are masked from 'package:stats':
## 
##     filter, lag
\end{verbatim}

\begin{verbatim}
## The following objects are masked from 'package:base':
## 
##     intersect, setdiff, setequal, union
\end{verbatim}

\begin{Shaded}
\begin{Highlighting}[]
\NormalTok{exp1 <-}\StringTok{ }\KeywordTok{read.csv}\NormalTok{(}\StringTok{"exp1.csv"}\NormalTok{, }\DataTypeTok{header =} \OtherTok{TRUE}\NormalTok{)}
\NormalTok{exp1a <-}\StringTok{ }\KeywordTok{read.csv}\NormalTok{(}\StringTok{"exp1a.csv"}\NormalTok{, }\DataTypeTok{header =} \OtherTok{TRUE}\NormalTok{)}
\NormalTok{exp1b <-}\StringTok{ }\KeywordTok{read.csv}\NormalTok{(}\StringTok{"exp1b.csv"}\NormalTok{, }\DataTypeTok{header =} \OtherTok{TRUE}\NormalTok{)}
\NormalTok{exp2 <-}\StringTok{ }\KeywordTok{read.csv}\NormalTok{(}\StringTok{"exp2.csv"}\NormalTok{, }\DataTypeTok{header =} \OtherTok{TRUE}\NormalTok{)}
\NormalTok{exp2a <-}\StringTok{ }\KeywordTok{read.csv}\NormalTok{(}\StringTok{"exp2a.csv"}\NormalTok{, }\DataTypeTok{header =} \OtherTok{TRUE}\NormalTok{)}
\NormalTok{exp1 <-}\StringTok{ }\KeywordTok{mutate}\NormalTok{(exp1,}\DataTypeTok{daycent=}\NormalTok{(DAY}\OperatorTok{-}\KeywordTok{mean}\NormalTok{(DAY)}\OperatorTok{/}\KeywordTok{sd}\NormalTok{(DAY)))}
\NormalTok{exp1a <-}\StringTok{ }\KeywordTok{mutate}\NormalTok{(exp1a,}\DataTypeTok{daycent=}\NormalTok{(DAY}\OperatorTok{-}\KeywordTok{mean}\NormalTok{(DAY)}\OperatorTok{/}\KeywordTok{sd}\NormalTok{(DAY)))}
\NormalTok{exp1b <-}\StringTok{ }\KeywordTok{mutate}\NormalTok{(exp1b,}\DataTypeTok{daycent=}\NormalTok{(DAY}\OperatorTok{-}\KeywordTok{mean}\NormalTok{(DAY)}\OperatorTok{/}\KeywordTok{sd}\NormalTok{(DAY)))}
\NormalTok{exp2 <-}\StringTok{ }\KeywordTok{mutate}\NormalTok{(exp2,}\DataTypeTok{daycent=}\NormalTok{(DAY}\OperatorTok{-}\KeywordTok{mean}\NormalTok{(DAY)}\OperatorTok{/}\KeywordTok{sd}\NormalTok{(DAY)))}
\NormalTok{exp2a <-}\StringTok{ }\KeywordTok{mutate}\NormalTok{(exp2a,}\DataTypeTok{daycent=}\NormalTok{(DAY}\OperatorTok{-}\KeywordTok{mean}\NormalTok{(DAY)}\OperatorTok{/}\KeywordTok{sd}\NormalTok{(DAY)))}
\end{Highlighting}
\end{Shaded}

I also wanted to see what would happen if I took out the lowest
performing frogs in each trial - the exp1a and exp2a datasets represent
have three subjects (5, 13, and 24) and one subject (9) removed from
them, respectively. The exp1b file has subjects 5 and 24 removed.
Attempting to remove outliers who didn't learn seemed to make the models
fit worse, so I've decided to use the original data with all subjects
included after all.

Now that I have my data uploaded\ldots{}I first needed to decided
whether or not to use the random slope model or the random intercept
model for my particular random effects. This was done with SUCCESS data
first\ldots{}

\begin{Shaded}
\begin{Highlighting}[]
\KeywordTok{library}\NormalTok{(lme4)}
\end{Highlighting}
\end{Shaded}

\begin{verbatim}
## Loading required package: Matrix
\end{verbatim}

\begin{Shaded}
\begin{Highlighting}[]
\NormalTok{SUCCESS1glmmfullrandomslope <-}\StringTok{ }\KeywordTok{glmer}\NormalTok{(}\KeywordTok{cbind}\NormalTok{(SUCCESS,}\DecValTok{3}\OperatorTok{-}\NormalTok{SUCCESS) }\OperatorTok{~}\StringTok{ }\NormalTok{daycent }\OperatorTok{*}\StringTok{ }\NormalTok{SEX }\OperatorTok{+}\StringTok{ }\NormalTok{(daycent}\OperatorTok{|}\NormalTok{SUB), }\DataTypeTok{data =}\NormalTok{ exp1,}\DataTypeTok{family=}\NormalTok{binomial)}
\NormalTok{SUCCESS1glmmfull <-}\StringTok{ }\KeywordTok{glmer}\NormalTok{(}\KeywordTok{cbind}\NormalTok{(SUCCESS,}\DecValTok{3}\OperatorTok{-}\NormalTok{SUCCESS) }\OperatorTok{~}\StringTok{ }\NormalTok{daycent }\OperatorTok{*}\StringTok{ }\NormalTok{SEX }\OperatorTok{+}\StringTok{ }\NormalTok{(}\DecValTok{1}\OperatorTok{|}\NormalTok{SUB), }\DataTypeTok{data =}\NormalTok{ exp1,}\DataTypeTok{family=}\NormalTok{binomial)}
\KeywordTok{anova}\NormalTok{(SUCCESS1glmmfullrandomslope,SUCCESS1glmmfull)}
\end{Highlighting}
\end{Shaded}

\begin{verbatim}
## Data: exp1
## Models:
## SUCCESS1glmmfull: cbind(SUCCESS, 3 - SUCCESS) ~ daycent * SEX + (1 | SUB)
## SUCCESS1glmmfullrandomslope: cbind(SUCCESS, 3 - SUCCESS) ~ daycent * SEX + (daycent | SUB)
##                             Df    AIC    BIC  logLik deviance Chisq Chi Df
## SUCCESS1glmmfull             5 620.63 638.03 -305.31   610.63             
## SUCCESS1glmmfullrandomslope  7 623.63 647.99 -304.81   609.63     1      2
##                             Pr(>Chisq)
## SUCCESS1glmmfull                      
## SUCCESS1glmmfullrandomslope     0.6065
\end{verbatim}

\begin{Shaded}
\begin{Highlighting}[]
\NormalTok{SUCCESS2glmmfullrandomslope <-}\StringTok{ }\KeywordTok{glmer}\NormalTok{(}\KeywordTok{cbind}\NormalTok{(SUCCESS,}\DecValTok{3}\OperatorTok{-}\NormalTok{SUCCESS) }\OperatorTok{~}\StringTok{ }\NormalTok{daycent }\OperatorTok{*}\StringTok{ }\NormalTok{SEX }\OperatorTok{+}\StringTok{ }\NormalTok{(daycent}\OperatorTok{|}\NormalTok{SUB), }\DataTypeTok{data =}\NormalTok{ exp2,}\DataTypeTok{family=}\NormalTok{binomial)}
\NormalTok{SUCCESS2glmmfull <-}\StringTok{ }\KeywordTok{glmer}\NormalTok{(}\KeywordTok{cbind}\NormalTok{(SUCCESS,}\DecValTok{3}\OperatorTok{-}\NormalTok{SUCCESS) }\OperatorTok{~}\StringTok{ }\NormalTok{daycent }\OperatorTok{*}\StringTok{ }\NormalTok{SEX }\OperatorTok{+}\StringTok{ }\NormalTok{(}\DecValTok{1}\OperatorTok{|}\NormalTok{SUB), }\DataTypeTok{data =}\NormalTok{ exp2,}\DataTypeTok{family=}\NormalTok{binomial)}
\KeywordTok{anova}\NormalTok{(SUCCESS2glmmfullrandomslope,SUCCESS2glmmfull)}
\end{Highlighting}
\end{Shaded}

\begin{verbatim}
## Data: exp2
## Models:
## SUCCESS2glmmfull: cbind(SUCCESS, 3 - SUCCESS) ~ daycent * SEX + (1 | SUB)
## SUCCESS2glmmfullrandomslope: cbind(SUCCESS, 3 - SUCCESS) ~ daycent * SEX + (daycent | SUB)
##                             Df    AIC    BIC  logLik deviance  Chisq
## SUCCESS2glmmfull             5 292.59 306.52 -141.29   282.59       
## SUCCESS2glmmfullrandomslope  7 286.42 305.93 -136.21   272.42 10.169
##                             Chi Df Pr(>Chisq)   
## SUCCESS2glmmfull                                
## SUCCESS2glmmfullrandomslope      2   0.006192 **
## ---
## Signif. codes:  0 '***' 0.001 '**' 0.01 '*' 0.05 '.' 0.1 ' ' 1
\end{verbatim}

Then, I did the same thing for my non-contingent errors\ldots{}

\begin{Shaded}
\begin{Highlighting}[]
\NormalTok{NC1glmmfullrandomslope <-}\StringTok{ }\KeywordTok{glmer}\NormalTok{(}\KeywordTok{cbind}\NormalTok{(NC_errors,}\DecValTok{3}\OperatorTok{-}\NormalTok{NC_errors) }\OperatorTok{~}\StringTok{ }\NormalTok{daycent }\OperatorTok{*}\StringTok{ }\NormalTok{SEX }\OperatorTok{+}\StringTok{ }\NormalTok{(daycent}\OperatorTok{|}\NormalTok{SUB), }\DataTypeTok{data =}\NormalTok{ exp1,}\DataTypeTok{family=}\NormalTok{binomial)}
\NormalTok{NC1glmmfull <-}\StringTok{ }\KeywordTok{glmer}\NormalTok{(}\KeywordTok{cbind}\NormalTok{(NC_errors,}\DecValTok{3}\OperatorTok{-}\NormalTok{NC_errors) }\OperatorTok{~}\StringTok{ }\NormalTok{daycent }\OperatorTok{*}\StringTok{ }\NormalTok{SEX }\OperatorTok{+}\StringTok{ }\NormalTok{(}\DecValTok{1}\OperatorTok{|}\NormalTok{SUB), }\DataTypeTok{data =}\NormalTok{ exp1,}\DataTypeTok{family=}\NormalTok{binomial)}
\KeywordTok{anova}\NormalTok{(NC1glmmfullrandomslope,NC1glmmfull)}
\end{Highlighting}
\end{Shaded}

\begin{verbatim}
## Data: exp1
## Models:
## NC1glmmfull: cbind(NC_errors, 3 - NC_errors) ~ daycent * SEX + (1 | SUB)
## NC1glmmfullrandomslope: cbind(NC_errors, 3 - NC_errors) ~ daycent * SEX + (daycent | 
## NC1glmmfullrandomslope:     SUB)
##                        Df    AIC   BIC  logLik deviance  Chisq Chi Df
## NC1glmmfull             5 530.80 548.2 -260.40   520.80              
## NC1glmmfullrandomslope  7 534.33 558.7 -260.17   520.33 0.4648      2
##                        Pr(>Chisq)
## NC1glmmfull                      
## NC1glmmfullrandomslope     0.7926
\end{verbatim}

\begin{Shaded}
\begin{Highlighting}[]
\NormalTok{NC2glmmfullrandomslope <-}\StringTok{ }\KeywordTok{glmer}\NormalTok{(}\KeywordTok{cbind}\NormalTok{(NC_errors,}\DecValTok{3}\OperatorTok{-}\NormalTok{NC_errors) }\OperatorTok{~}\StringTok{ }\NormalTok{daycent }\OperatorTok{*}\StringTok{ }\NormalTok{SEX }\OperatorTok{+}\StringTok{ }\NormalTok{(daycent}\OperatorTok{|}\NormalTok{SUB), }\DataTypeTok{data =}\NormalTok{ exp2,}\DataTypeTok{family=}\NormalTok{binomial)}
\NormalTok{NC2glmmfull <-}\StringTok{ }\KeywordTok{glmer}\NormalTok{(}\KeywordTok{cbind}\NormalTok{(NC_errors,}\DecValTok{3}\OperatorTok{-}\NormalTok{NC_errors) }\OperatorTok{~}\StringTok{ }\NormalTok{daycent }\OperatorTok{*}\StringTok{ }\NormalTok{SEX }\OperatorTok{+}\StringTok{ }\NormalTok{(}\DecValTok{1}\OperatorTok{|}\NormalTok{SUB), }\DataTypeTok{data =}\NormalTok{ exp2,}\DataTypeTok{family=}\NormalTok{binomial)}
\KeywordTok{anova}\NormalTok{(NC2glmmfullrandomslope,NC2glmmfull)}
\end{Highlighting}
\end{Shaded}

\begin{verbatim}
## Data: exp2
## Models:
## NC2glmmfull: cbind(NC_errors, 3 - NC_errors) ~ daycent * SEX + (1 | SUB)
## NC2glmmfullrandomslope: cbind(NC_errors, 3 - NC_errors) ~ daycent * SEX + (daycent | 
## NC2glmmfullrandomslope:     SUB)
##                        Df    AIC    BIC  logLik deviance  Chisq Chi Df
## NC2glmmfull             5 206.93 220.86 -98.462   196.93              
## NC2glmmfullrandomslope  7 210.69 230.20 -98.345   196.69 0.2354      2
##                        Pr(>Chisq)
## NC2glmmfull                      
## NC2glmmfullrandomslope     0.8889
\end{verbatim}

With the AIC and BIC values so close to each other for each model, these
analyses determined that it is sufficient to use random intercepts for
my random effects (1 \textbar{} SUB) for both SUCCESS and NC\_errors to
make the models a bit simpler. Lastly, I want to run this test for my
third dependent variable of interest, P\_errors\ldots{}

\subsubsection{however, P\_errors can not be run using the binomal error
model that was used for SUCCESS and NC\_errors, so I used the poisson
model, which is better for various integers (poisson would be better to
use than gamma in this situation,
correct?)}\label{however-p_errors-can-not-be-run-using-the-binomal-error-model-that-was-used-for-success-and-nc_errors-so-i-used-the-poisson-model-which-is-better-for-various-integers-poisson-would-be-better-to-use-than-gamma-in-this-situation-correct}

\begin{Shaded}
\begin{Highlighting}[]
\NormalTok{P1glmmfullranslope <-}\StringTok{ }\KeywordTok{glmer}\NormalTok{(P_errors }\OperatorTok{~}\StringTok{ }\NormalTok{daycent }\OperatorTok{*}\StringTok{ }\NormalTok{SEX }\OperatorTok{+}\StringTok{ }\NormalTok{(daycent}\OperatorTok{|}\NormalTok{SUB), }\DataTypeTok{data =}\NormalTok{ exp1,}\DataTypeTok{family=}\NormalTok{poisson)}
\NormalTok{P1glmmfull <-}\StringTok{ }\KeywordTok{glmer}\NormalTok{(P_errors }\OperatorTok{~}\StringTok{ }\NormalTok{daycent }\OperatorTok{*}\StringTok{ }\NormalTok{SEX }\OperatorTok{+}\StringTok{ }\NormalTok{(}\DecValTok{1}\OperatorTok{|}\NormalTok{SUB), }\DataTypeTok{data =}\NormalTok{ exp1,}\DataTypeTok{family=}\NormalTok{poisson)}
\end{Highlighting}
\end{Shaded}

\begin{verbatim}
## Warning in checkConv(attr(opt, "derivs"), opt$par, ctrl = control
## $checkConv, : Model failed to converge with max|grad| = 0.0011556 (tol =
## 0.001, component 1)
\end{verbatim}

\begin{Shaded}
\begin{Highlighting}[]
\KeywordTok{anova}\NormalTok{(P1glmmfullranslope,P1glmmfull)}
\end{Highlighting}
\end{Shaded}

\begin{verbatim}
## Data: exp1
## Models:
## P1glmmfull: P_errors ~ daycent * SEX + (1 | SUB)
## P1glmmfullranslope: P_errors ~ daycent * SEX + (daycent | SUB)
##                    Df    AIC    BIC  logLik deviance  Chisq Chi Df
## P1glmmfull          5 872.99 890.40 -431.50   862.99              
## P1glmmfullranslope  7 866.86 891.22 -426.43   852.86 10.136      2
##                    Pr(>Chisq)   
## P1glmmfull                      
## P1glmmfullranslope   0.006294 **
## ---
## Signif. codes:  0 '***' 0.001 '**' 0.01 '*' 0.05 '.' 0.1 ' ' 1
\end{verbatim}

\begin{Shaded}
\begin{Highlighting}[]
\NormalTok{P2glmmfullranslope <-}\StringTok{ }\KeywordTok{glmer}\NormalTok{(P_errors }\OperatorTok{~}\StringTok{ }\NormalTok{daycent }\OperatorTok{*}\StringTok{ }\NormalTok{SEX }\OperatorTok{+}\StringTok{ }\NormalTok{(daycent}\OperatorTok{|}\NormalTok{SUB), }\DataTypeTok{data =}\NormalTok{ exp2,}\DataTypeTok{family=}\NormalTok{poisson)}
\NormalTok{P2glmmfull <-}\StringTok{ }\KeywordTok{glmer}\NormalTok{(P_errors }\OperatorTok{~}\StringTok{ }\NormalTok{daycent }\OperatorTok{*}\StringTok{ }\NormalTok{SEX }\OperatorTok{+}\StringTok{ }\NormalTok{(}\DecValTok{1}\OperatorTok{|}\NormalTok{SUB), }\DataTypeTok{data =}\NormalTok{ exp2,}\DataTypeTok{family=}\NormalTok{poisson)}
\KeywordTok{anova}\NormalTok{(P2glmmfullranslope,P2glmmfull)}
\end{Highlighting}
\end{Shaded}

\begin{verbatim}
## Data: exp2
## Models:
## P2glmmfull: P_errors ~ daycent * SEX + (1 | SUB)
## P2glmmfullranslope: P_errors ~ daycent * SEX + (daycent | SUB)
##                    Df    AIC    BIC  logLik deviance  Chisq Chi Df
## P2glmmfull          5 371.79 385.73 -180.90   361.79              
## P2glmmfullranslope  7 369.13 388.64 -177.56   355.13 6.6653      2
##                    Pr(>Chisq)  
## P2glmmfull                     
## P2glmmfullranslope     0.0357 *
## ---
## Signif. codes:  0 '***' 0.001 '**' 0.01 '*' 0.05 '.' 0.1 ' ' 1
\end{verbatim}

\begin{Shaded}
\begin{Highlighting}[]
\NormalTok{P1aglmmfull <-}\StringTok{ }\KeywordTok{glmer}\NormalTok{(P_errors }\OperatorTok{~}\StringTok{ }\NormalTok{daycent }\OperatorTok{*}\StringTok{ }\NormalTok{SEX }\OperatorTok{+}\StringTok{ }\NormalTok{(}\DecValTok{1}\OperatorTok{|}\NormalTok{SUB), }\DataTypeTok{data =}\NormalTok{ exp1a,}\DataTypeTok{family=}\NormalTok{poisson)}
\NormalTok{P2aglmmfull <-}\StringTok{ }\KeywordTok{glmer}\NormalTok{(P_errors }\OperatorTok{~}\StringTok{ }\NormalTok{daycent }\OperatorTok{*}\StringTok{ }\NormalTok{SEX }\OperatorTok{+}\StringTok{ }\NormalTok{(}\DecValTok{1}\OperatorTok{|}\NormalTok{SUB), }\DataTypeTok{data =}\NormalTok{ exp2a,}\DataTypeTok{family=}\NormalTok{poisson)}
\end{Highlighting}
\end{Shaded}

Given how close the AIC and BIC values are for both, I believe this
means that the fit is fairly equal (and bad\ldots{}) for each model,
depsite the p value being much smaller than it was for the first two
models. Not quite sure what this means, but I think it means that these
two models are VERY different from each other.

\subsubsection{Now, I can compare all of the models! My full model looks
at the INTERACTION of day and sex; another model (add) treats sex as a
fixed effect independent of day; a third model removes sex completely
and looks as how success varies with day (day); and lastly a null model
removes all covariates to examine the effect of their presence
(null.}\label{now-i-can-compare-all-of-the-models-my-full-model-looks-at-the-interaction-of-day-and-sex-another-model-add-treats-sex-as-a-fixed-effect-independent-of-day-a-third-model-removes-sex-completely-and-looks-as-how-success-varies-with-day-day-and-lastly-a-null-model-removes-all-covariates-to-examine-the-effect-of-their-presence-null.}

\begin{Shaded}
\begin{Highlighting}[]
\NormalTok{SUCCESS1glmmadd <-}\StringTok{ }\KeywordTok{glmer}\NormalTok{(}\KeywordTok{cbind}\NormalTok{(SUCCESS,}\DecValTok{3}\OperatorTok{-}\NormalTok{SUCCESS) }\OperatorTok{~}\StringTok{ }\NormalTok{daycent }\OperatorTok{+}\StringTok{ }\NormalTok{SEX }\OperatorTok{+}\StringTok{ }\NormalTok{(}\DecValTok{1}\OperatorTok{|}\NormalTok{SUB), }\DataTypeTok{data =}\NormalTok{ exp1,}\DataTypeTok{family=}\NormalTok{binomial)}
\NormalTok{SUCCESS1glmmday <-}\StringTok{ }\KeywordTok{glmer}\NormalTok{(}\KeywordTok{cbind}\NormalTok{(SUCCESS,}\DecValTok{3}\OperatorTok{-}\NormalTok{SUCCESS) }\OperatorTok{~}\StringTok{ }\NormalTok{daycent }\OperatorTok{+}\StringTok{ }\NormalTok{(}\DecValTok{1}\OperatorTok{|}\NormalTok{SUB), }\DataTypeTok{data =}\NormalTok{ exp1,}\DataTypeTok{family=}\NormalTok{binomial)}
\NormalTok{SUCCESS1glmmnull <-}\StringTok{ }\KeywordTok{glmer}\NormalTok{(}\KeywordTok{cbind}\NormalTok{(SUCCESS,}\DecValTok{3}\OperatorTok{-}\NormalTok{SUCCESS) }\OperatorTok{~}\StringTok{ }\NormalTok{(}\DecValTok{1}\OperatorTok{|}\NormalTok{SUB), }\DataTypeTok{data =}\NormalTok{ exp1,}\DataTypeTok{family=}\NormalTok{binomial)}

\KeywordTok{anova}\NormalTok{(SUCCESS1glmmfull, SUCCESS1glmmadd, SUCCESS1glmmday, SUCCESS1glmmnull, }\DataTypeTok{test=}\StringTok{"Chisq"}\NormalTok{)}
\end{Highlighting}
\end{Shaded}

\begin{verbatim}
## Data: exp1
## Models:
## SUCCESS1glmmnull: cbind(SUCCESS, 3 - SUCCESS) ~ (1 | SUB)
## SUCCESS1glmmday: cbind(SUCCESS, 3 - SUCCESS) ~ daycent + (1 | SUB)
## SUCCESS1glmmadd: cbind(SUCCESS, 3 - SUCCESS) ~ daycent + SEX + (1 | SUB)
## SUCCESS1glmmfull: cbind(SUCCESS, 3 - SUCCESS) ~ daycent * SEX + (1 | SUB)
##                  Df    AIC    BIC  logLik deviance   Chisq Chi Df
## SUCCESS1glmmnull  2 663.41 670.37 -329.70   659.41               
## SUCCESS1glmmday   3 620.50 630.94 -307.25   614.50 44.9128      1
## SUCCESS1glmmadd   4 622.26 636.18 -307.13   614.26  0.2374      1
## SUCCESS1glmmfull  5 620.63 638.03 -305.31   610.63  3.6317      1
##                  Pr(>Chisq)    
## SUCCESS1glmmnull               
## SUCCESS1glmmday    2.06e-11 ***
## SUCCESS1glmmadd     0.62613    
## SUCCESS1glmmfull    0.05669 .  
## ---
## Signif. codes:  0 '***' 0.001 '**' 0.01 '*' 0.05 '.' 0.1 ' ' 1
\end{verbatim}

This shows some small evidence that there is an interaction between sex
and day (frogs from different sexes have different trends in their
abilities). The p-value here is .057, which is borderline significant

\begin{Shaded}
\begin{Highlighting}[]
\NormalTok{SUCCESS2glmmadd <-}\StringTok{ }\KeywordTok{glmer}\NormalTok{(}\KeywordTok{cbind}\NormalTok{(SUCCESS,}\DecValTok{3}\OperatorTok{-}\NormalTok{SUCCESS) }\OperatorTok{~}\StringTok{ }\NormalTok{daycent }\OperatorTok{+}\StringTok{ }\NormalTok{SEX }\OperatorTok{+}\StringTok{ }\NormalTok{(}\DecValTok{1}\OperatorTok{|}\NormalTok{SUB), }\DataTypeTok{data =}\NormalTok{ exp2,}\DataTypeTok{family=}\NormalTok{binomial)}
\NormalTok{SUCCESS2glmmday <-}\StringTok{ }\KeywordTok{glmer}\NormalTok{(}\KeywordTok{cbind}\NormalTok{(SUCCESS,}\DecValTok{3}\OperatorTok{-}\NormalTok{SUCCESS) }\OperatorTok{~}\StringTok{ }\NormalTok{daycent }\OperatorTok{+}\StringTok{ }\NormalTok{(}\DecValTok{1}\OperatorTok{|}\NormalTok{SUB), }\DataTypeTok{data =}\NormalTok{ exp2,}\DataTypeTok{family=}\NormalTok{binomial)}
\NormalTok{SUCCESS2glmmnull <-}\StringTok{ }\KeywordTok{glmer}\NormalTok{(}\KeywordTok{cbind}\NormalTok{(SUCCESS,}\DecValTok{3}\OperatorTok{-}\NormalTok{SUCCESS) }\OperatorTok{~}\StringTok{ }\NormalTok{(}\DecValTok{1}\OperatorTok{|}\NormalTok{SUB), }\DataTypeTok{data =}\NormalTok{ exp2,}\DataTypeTok{family=}\NormalTok{binomial)}

\KeywordTok{anova}\NormalTok{(SUCCESS2glmmfull, SUCCESS2glmmadd, SUCCESS2glmmday, SUCCESS2glmmnull, }\DataTypeTok{test=}\StringTok{"Chisq"}\NormalTok{)}
\end{Highlighting}
\end{Shaded}

\begin{verbatim}
## Data: exp2
## Models:
## SUCCESS2glmmnull: cbind(SUCCESS, 3 - SUCCESS) ~ (1 | SUB)
## SUCCESS2glmmday: cbind(SUCCESS, 3 - SUCCESS) ~ daycent + (1 | SUB)
## SUCCESS2glmmadd: cbind(SUCCESS, 3 - SUCCESS) ~ daycent + SEX + (1 | SUB)
## SUCCESS2glmmfull: cbind(SUCCESS, 3 - SUCCESS) ~ daycent * SEX + (1 | SUB)
##                  Df    AIC    BIC  logLik deviance   Chisq Chi Df
## SUCCESS2glmmnull  2 304.70 310.28 -150.35   300.70               
## SUCCESS2glmmday   3 288.88 297.25 -141.44   282.88 17.8190      1
## SUCCESS2glmmadd   4 290.88 302.02 -141.44   282.88  0.0078      1
## SUCCESS2glmmfull  5 292.59 306.52 -141.29   282.59  0.2886      1
##                  Pr(>Chisq)    
## SUCCESS2glmmnull               
## SUCCESS2glmmday   2.429e-05 ***
## SUCCESS2glmmadd      0.9298    
## SUCCESS2glmmfull     0.5911    
## ---
## Signif. codes:  0 '***' 0.001 '**' 0.01 '*' 0.05 '.' 0.1 ' ' 1
\end{verbatim}

This result shows no evidence for their being an interaction between sex
and day for my second experiment.

\subsubsection{Now to do the same thing for
NC\_errors\ldots{}}\label{now-to-do-the-same-thing-for-nc_errors}

\begin{Shaded}
\begin{Highlighting}[]
\NormalTok{NC1glmmadd <-}\StringTok{ }\KeywordTok{glmer}\NormalTok{(}\KeywordTok{cbind}\NormalTok{(NC_errors,}\DecValTok{3}\OperatorTok{-}\NormalTok{NC_errors) }\OperatorTok{~}\StringTok{ }\NormalTok{daycent }\OperatorTok{+}\StringTok{ }\NormalTok{SEX }\OperatorTok{+}\StringTok{ }\NormalTok{(}\DecValTok{1}\OperatorTok{|}\NormalTok{SUB), }\DataTypeTok{data =}\NormalTok{ exp1,}\DataTypeTok{family=}\NormalTok{binomial)}
\NormalTok{NC1glmmday <-}\StringTok{ }\KeywordTok{glmer}\NormalTok{(}\KeywordTok{cbind}\NormalTok{(NC_errors,}\DecValTok{3}\OperatorTok{-}\NormalTok{NC_errors) }\OperatorTok{~}\StringTok{ }\NormalTok{daycent }\OperatorTok{+}\StringTok{ }\NormalTok{(}\DecValTok{1}\OperatorTok{|}\NormalTok{SUB), }\DataTypeTok{data =}\NormalTok{ exp1,}\DataTypeTok{family=}\NormalTok{binomial)}
\NormalTok{NC1glmmnull <-}\StringTok{ }\KeywordTok{glmer}\NormalTok{(}\KeywordTok{cbind}\NormalTok{(NC_errors,}\DecValTok{3}\OperatorTok{-}\NormalTok{NC_errors) }\OperatorTok{~}\StringTok{ }\NormalTok{(}\DecValTok{1}\OperatorTok{|}\NormalTok{SUB), }\DataTypeTok{data =}\NormalTok{ exp1,}\DataTypeTok{family=}\NormalTok{binomial)}

\KeywordTok{anova}\NormalTok{(NC1glmmfull, NC1glmmadd, NC1glmmday, NC1glmmnull, }\DataTypeTok{test=}\StringTok{"Chisq"}\NormalTok{)}
\end{Highlighting}
\end{Shaded}

\begin{verbatim}
## Data: exp1
## Models:
## NC1glmmnull: cbind(NC_errors, 3 - NC_errors) ~ (1 | SUB)
## NC1glmmday: cbind(NC_errors, 3 - NC_errors) ~ daycent + (1 | SUB)
## NC1glmmadd: cbind(NC_errors, 3 - NC_errors) ~ daycent + SEX + (1 | SUB)
## NC1glmmfull: cbind(NC_errors, 3 - NC_errors) ~ daycent * SEX + (1 | SUB)
##             Df    AIC    BIC  logLik deviance   Chisq Chi Df Pr(>Chisq)
## NC1glmmnull  2 562.20 569.16 -279.10   558.20                          
## NC1glmmday   3 528.57 539.01 -261.29   522.57 35.6317      1  2.384e-09
## NC1glmmadd   4 530.56 544.49 -261.28   522.56  0.0075      1     0.9308
## NC1glmmfull  5 530.80 548.20 -260.40   520.80  1.7662      1     0.1839
##                
## NC1glmmnull    
## NC1glmmday  ***
## NC1glmmadd     
## NC1glmmfull    
## ---
## Signif. codes:  0 '***' 0.001 '**' 0.01 '*' 0.05 '.' 0.1 ' ' 1
\end{verbatim}

\begin{Shaded}
\begin{Highlighting}[]
\NormalTok{NC2glmmadd <-}\StringTok{ }\KeywordTok{glmer}\NormalTok{(}\KeywordTok{cbind}\NormalTok{(NC_errors,}\DecValTok{3}\OperatorTok{-}\NormalTok{NC_errors) }\OperatorTok{~}\StringTok{ }\NormalTok{daycent }\OperatorTok{+}\StringTok{ }\NormalTok{SEX }\OperatorTok{+}\StringTok{ }\NormalTok{(}\DecValTok{1}\OperatorTok{|}\NormalTok{SUB), }\DataTypeTok{data =}\NormalTok{ exp2,}\DataTypeTok{family=}\NormalTok{binomial)}
\NormalTok{NC2glmmday <-}\StringTok{ }\KeywordTok{glmer}\NormalTok{(}\KeywordTok{cbind}\NormalTok{(NC_errors,}\DecValTok{3}\OperatorTok{-}\NormalTok{NC_errors) }\OperatorTok{~}\StringTok{ }\NormalTok{daycent }\OperatorTok{+}\StringTok{ }\NormalTok{(}\DecValTok{1}\OperatorTok{|}\NormalTok{SUB), }\DataTypeTok{data =}\NormalTok{ exp2,}\DataTypeTok{family=}\NormalTok{binomial)}
\NormalTok{NC2glmmnull <-}\StringTok{ }\KeywordTok{glmer}\NormalTok{(}\KeywordTok{cbind}\NormalTok{(NC_errors,}\DecValTok{3}\OperatorTok{-}\NormalTok{NC_errors) }\OperatorTok{~}\StringTok{ }\NormalTok{(}\DecValTok{1}\OperatorTok{|}\NormalTok{SUB), }\DataTypeTok{data =}\NormalTok{ exp2,}\DataTypeTok{family=}\NormalTok{binomial)}

\KeywordTok{anova}\NormalTok{(NC2glmmfull, NC2glmmadd, NC2glmmday, NC2glmmnull, }\DataTypeTok{test=}\StringTok{"Chisq"}\NormalTok{)}
\end{Highlighting}
\end{Shaded}

\begin{verbatim}
## Data: exp2
## Models:
## NC2glmmnull: cbind(NC_errors, 3 - NC_errors) ~ (1 | SUB)
## NC2glmmday: cbind(NC_errors, 3 - NC_errors) ~ daycent + (1 | SUB)
## NC2glmmadd: cbind(NC_errors, 3 - NC_errors) ~ daycent + SEX + (1 | SUB)
## NC2glmmfull: cbind(NC_errors, 3 - NC_errors) ~ daycent * SEX + (1 | SUB)
##             Df    AIC    BIC   logLik deviance   Chisq Chi Df Pr(>Chisq)
## NC2glmmnull  2 253.00 258.58 -124.500   249.00                          
## NC2glmmday   3 203.91 212.28  -98.956   197.91 51.0888      1  8.828e-13
## NC2glmmadd   4 205.39 216.54  -98.696   197.39  0.5210      1     0.4704
## NC2glmmfull  5 206.93 220.86  -98.462   196.93  0.4665      1     0.4946
##                
## NC2glmmnull    
## NC2glmmday  ***
## NC2glmmadd     
## NC2glmmfull    
## ---
## Signif. codes:  0 '***' 0.001 '**' 0.01 '*' 0.05 '.' 0.1 ' ' 1
\end{verbatim}

Both of these analyses suggest that there is no interaction between sex
and NC\_errors.

\subsubsection{Lastly, P\_errors}\label{lastly-p_errors}

\begin{Shaded}
\begin{Highlighting}[]
\NormalTok{P1glmmadd <-}\StringTok{ }\KeywordTok{glmer}\NormalTok{(P_errors }\OperatorTok{~}\StringTok{ }\NormalTok{daycent }\OperatorTok{+}\StringTok{ }\NormalTok{SEX }\OperatorTok{+}\StringTok{ }\NormalTok{(}\DecValTok{1}\OperatorTok{|}\NormalTok{SUB), }\DataTypeTok{data =}\NormalTok{ exp1,}\DataTypeTok{family=}\NormalTok{poisson)}
\NormalTok{P1glmmday <-}\StringTok{ }\KeywordTok{glmer}\NormalTok{(P_errors }\OperatorTok{~}\StringTok{ }\NormalTok{daycent }\OperatorTok{+}\StringTok{ }\NormalTok{(}\DecValTok{1}\OperatorTok{|}\NormalTok{SUB), }\DataTypeTok{data =}\NormalTok{ exp1,}\DataTypeTok{family=}\NormalTok{poisson)}
\NormalTok{P1glmmnull <-}\StringTok{ }\KeywordTok{glmer}\NormalTok{(P_errors }\OperatorTok{~}\StringTok{ }\NormalTok{(}\DecValTok{1}\OperatorTok{|}\NormalTok{SUB), }\DataTypeTok{data =}\NormalTok{ exp1,}\DataTypeTok{family=}\NormalTok{poisson)}

\KeywordTok{anova}\NormalTok{(P1glmmfull, P1glmmadd, P1glmmday, P1glmmnull, }\DataTypeTok{test=}\StringTok{"Chisq"}\NormalTok{)}
\end{Highlighting}
\end{Shaded}

\begin{verbatim}
## Data: exp1
## Models:
## P1glmmnull: P_errors ~ (1 | SUB)
## P1glmmday: P_errors ~ daycent + (1 | SUB)
## P1glmmadd: P_errors ~ daycent + SEX + (1 | SUB)
## P1glmmfull: P_errors ~ daycent * SEX + (1 | SUB)
##            Df    AIC    BIC  logLik deviance   Chisq Chi Df Pr(>Chisq)    
## P1glmmnull  2 878.91 885.88 -437.46   874.91                              
## P1glmmday   3 869.83 880.28 -431.92   863.83 11.0795      1  0.0008729 ***
## P1glmmadd   4 871.16 885.09 -431.58   863.16  0.6719      1  0.4123886    
## P1glmmfull  5 872.99 890.40 -431.50   862.99  0.1695      1  0.6805362    
## ---
## Signif. codes:  0 '***' 0.001 '**' 0.01 '*' 0.05 '.' 0.1 ' ' 1
\end{verbatim}

\begin{Shaded}
\begin{Highlighting}[]
\NormalTok{P2glmmadd <-}\StringTok{ }\KeywordTok{glmer}\NormalTok{(P_errors }\OperatorTok{~}\StringTok{ }\NormalTok{daycent }\OperatorTok{+}\StringTok{ }\NormalTok{SEX }\OperatorTok{+}\StringTok{ }\NormalTok{(}\DecValTok{1}\OperatorTok{|}\NormalTok{SUB), }\DataTypeTok{data =}\NormalTok{ exp2,}\DataTypeTok{family=}\NormalTok{poisson)}
\NormalTok{P2glmmday <-}\StringTok{ }\KeywordTok{glmer}\NormalTok{(P_errors }\OperatorTok{~}\StringTok{ }\NormalTok{daycent }\OperatorTok{+}\StringTok{ }\NormalTok{(}\DecValTok{1}\OperatorTok{|}\NormalTok{SUB), }\DataTypeTok{data =}\NormalTok{ exp2,}\DataTypeTok{family=}\NormalTok{poisson)}
\NormalTok{P2glmmnull <-}\StringTok{ }\KeywordTok{glmer}\NormalTok{(P_errors }\OperatorTok{~}\StringTok{ }\NormalTok{(}\DecValTok{1}\OperatorTok{|}\NormalTok{SUB), }\DataTypeTok{data =}\NormalTok{ exp2,}\DataTypeTok{family=}\NormalTok{poisson)}

\KeywordTok{anova}\NormalTok{(P2glmmfull, P2glmmadd, P2glmmday, P2glmmnull, }\DataTypeTok{test=}\StringTok{"Chisq"}\NormalTok{)}
\end{Highlighting}
\end{Shaded}

\begin{verbatim}
## Data: exp2
## Models:
## P2glmmnull: P_errors ~ (1 | SUB)
## P2glmmday: P_errors ~ daycent + (1 | SUB)
## P2glmmadd: P_errors ~ daycent + SEX + (1 | SUB)
## P2glmmfull: P_errors ~ daycent * SEX + (1 | SUB)
##            Df    AIC    BIC  logLik deviance  Chisq Chi Df Pr(>Chisq)   
## P2glmmnull  2 374.15 379.73 -185.08   370.15                            
## P2glmmday   3 369.11 377.47 -181.56   363.11 7.0449      1   0.007949 **
## P2glmmadd   4 369.92 381.07 -180.96   361.92 1.1914      1   0.275054   
## P2glmmfull  5 371.79 385.73 -180.90   361.79 0.1254      1   0.723227   
## ---
## Signif. codes:  0 '***' 0.001 '**' 0.01 '*' 0.05 '.' 0.1 ' ' 1
\end{verbatim}

This shows that there is also no interaction between sex and P\_error,
but that P\_errors vary significantly with the day for both experiments
- could I possibly use this to say that if P\_errors decrease
signficantly by day and if SUCCESS increases significantly by day, that
the frogs as a whole learned from this experiment??

\section{Another way we can try to understand what these models tell us
is by simply generating bootstrap confidence
intervals.}\label{another-way-we-can-try-to-understand-what-these-models-tell-us-is-by-simply-generating-bootstrap-confidence-intervals.}

\subsubsection{As the name implies, this process uses the computed model
data itself to estimate the variation of statistics - essentially using
the computed data to come up with confidence
intervals.}\label{as-the-name-implies-this-process-uses-the-computed-model-data-itself-to-estimate-the-variation-of-statistics---essentially-using-the-computed-data-to-come-up-with-confidence-intervals.}

\begin{Shaded}
\begin{Highlighting}[]
\KeywordTok{set.seed}\NormalTok{(}\DecValTok{4}\NormalTok{)}
\NormalTok{bootconf <-}\StringTok{ }\KeywordTok{confint}\NormalTok{(SUCCESS1glmmfull,}\DataTypeTok{method=}\StringTok{"boot"}\NormalTok{,}\DataTypeTok{nsim=}\DecValTok{2000}\NormalTok{)}
\end{Highlighting}
\end{Shaded}

\begin{verbatim}
## Computing bootstrap confidence intervals ...
\end{verbatim}

\begin{verbatim}
## Warning in checkConv(attr(opt, "derivs"), opt$par, ctrl = control
## $checkConv, : Model failed to converge with max|grad| = 0.0168774 (tol =
## 0.001, component 1)
\end{verbatim}

\begin{verbatim}
## Warning in checkConv(attr(opt, "derivs"), opt$par, ctrl = control
## $checkConv, : Model failed to converge with max|grad| = 0.0164901 (tol =
## 0.001, component 1)
\end{verbatim}

\begin{verbatim}
## Warning in checkConv(attr(opt, "derivs"), opt$par, ctrl = control
## $checkConv, : Model failed to converge with max|grad| = 0.0165298 (tol =
## 0.001, component 1)
\end{verbatim}

\begin{verbatim}
## Warning in checkConv(attr(opt, "derivs"), opt$par, ctrl = control
## $checkConv, : Model failed to converge with max|grad| = 0.021751 (tol =
## 0.001, component 1)
\end{verbatim}

\begin{verbatim}
## Warning in checkConv(attr(opt, "derivs"), opt$par, ctrl = control
## $checkConv, : Model failed to converge with max|grad| = 0.0155816 (tol =
## 0.001, component 1)
\end{verbatim}

\begin{verbatim}
## Warning in checkConv(attr(opt, "derivs"), opt$par, ctrl = control
## $checkConv, : Model failed to converge with max|grad| = 0.0211349 (tol =
## 0.001, component 1)
\end{verbatim}

\begin{verbatim}
## Warning in checkConv(attr(opt, "derivs"), opt$par, ctrl = control
## $checkConv, : Model failed to converge with max|grad| = 0.00105986 (tol =
## 0.001, component 1)
\end{verbatim}

\begin{verbatim}
## Warning in checkConv(attr(opt, "derivs"), opt$par, ctrl = control
## $checkConv, : Model failed to converge with max|grad| = 0.0223567 (tol =
## 0.001, component 1)
\end{verbatim}

\begin{verbatim}
## Warning in checkConv(attr(opt, "derivs"), opt$par, ctrl = control
## $checkConv, : Model failed to converge with max|grad| = 0.00386211 (tol =
## 0.001, component 1)
\end{verbatim}

\begin{verbatim}
## Warning in checkConv(attr(opt, "derivs"), opt$par, ctrl = control
## $checkConv, : Model failed to converge with max|grad| = 0.00100899 (tol =
## 0.001, component 1)
\end{verbatim}

\begin{verbatim}
## Warning in checkConv(attr(opt, "derivs"), opt$par, ctrl = control
## $checkConv, : Model failed to converge with max|grad| = 0.0177264 (tol =
## 0.001, component 1)
\end{verbatim}

\begin{Shaded}
\begin{Highlighting}[]
\NormalTok{bootconf}
\end{Highlighting}
\end{Shaded}

\begin{verbatim}
##                   2.5 %       97.5 %
## .sig01        0.3329639  0.866667710
## (Intercept)  -1.4695813 -0.393942245
## daycent       0.1658979  0.326569307
## SEXM         -0.1712585  1.263868904
## daycent:SEXM -0.2202503  0.007656257
\end{verbatim}

\begin{Shaded}
\begin{Highlighting}[]
\KeywordTok{set.seed}\NormalTok{(}\DecValTok{4}\NormalTok{)}
\NormalTok{bootconf <-}\StringTok{ }\KeywordTok{confint}\NormalTok{(SUCCESS2glmmfull,}\DataTypeTok{method=}\StringTok{"boot"}\NormalTok{,}\DataTypeTok{nsim=}\DecValTok{2000}\NormalTok{)}
\end{Highlighting}
\end{Shaded}

\begin{verbatim}
## Computing bootstrap confidence intervals ...
\end{verbatim}

\begin{verbatim}
## Warning in checkConv(attr(opt, "derivs"), opt$par, ctrl = control
## $checkConv, : Model failed to converge with max|grad| = 0.0159956 (tol =
## 0.001, component 1)
\end{verbatim}

\begin{verbatim}
## Warning in checkConv(attr(opt, "derivs"), opt$par, ctrl = control
## $checkConv, : unable to evaluate scaled gradient
\end{verbatim}

\begin{verbatim}
## Warning in checkConv(attr(opt, "derivs"), opt$par, ctrl = control
## $checkConv, : Model failed to converge: degenerate Hessian with 1 negative
## eigenvalues
\end{verbatim}

\begin{verbatim}
## Warning in checkConv(attr(opt, "derivs"), opt$par, ctrl = control
## $checkConv, : Model failed to converge with max|grad| = 0.0227025 (tol =
## 0.001, component 1)
\end{verbatim}

\begin{verbatim}
## Warning in checkConv(attr(opt, "derivs"), opt$par, ctrl = control
## $checkConv, : unable to evaluate scaled gradient
\end{verbatim}

\begin{verbatim}
## Warning in checkConv(attr(opt, "derivs"), opt$par, ctrl = control
## $checkConv, : Model failed to converge: degenerate Hessian with 1 negative
## eigenvalues
\end{verbatim}

\begin{verbatim}
## Warning in checkConv(attr(opt, "derivs"), opt$par, ctrl = control
## $checkConv, : unable to evaluate scaled gradient
\end{verbatim}

\begin{verbatim}
## Warning in checkConv(attr(opt, "derivs"), opt$par, ctrl = control
## $checkConv, : Model failed to converge: degenerate Hessian with 1 negative
## eigenvalues
\end{verbatim}

\begin{verbatim}
## Warning in checkConv(attr(opt, "derivs"), opt$par, ctrl = control
## $checkConv, : Model failed to converge with max|grad| = 0.0136334 (tol =
## 0.001, component 1)
\end{verbatim}

\begin{verbatim}
## Warning in checkConv(attr(opt, "derivs"), opt$par, ctrl = control
## $checkConv, : Model failed to converge with max|grad| = 0.0163831 (tol =
## 0.001, component 1)
\end{verbatim}

\begin{verbatim}
## Warning in checkConv(attr(opt, "derivs"), opt$par, ctrl = control
## $checkConv, : unable to evaluate scaled gradient
\end{verbatim}

\begin{verbatim}
## Warning in checkConv(attr(opt, "derivs"), opt$par, ctrl = control
## $checkConv, : Model failed to converge: degenerate Hessian with 1 negative
## eigenvalues
\end{verbatim}

\begin{verbatim}
## Warning in checkConv(attr(opt, "derivs"), opt$par, ctrl = control
## $checkConv, : unable to evaluate scaled gradient
\end{verbatim}

\begin{verbatim}
## Warning in checkConv(attr(opt, "derivs"), opt$par, ctrl = control
## $checkConv, : Model failed to converge: degenerate Hessian with 1 negative
## eigenvalues
\end{verbatim}

\begin{verbatim}
## Warning in checkConv(attr(opt, "derivs"), opt$par, ctrl = control
## $checkConv, : unable to evaluate scaled gradient
\end{verbatim}

\begin{verbatim}
## Warning in checkConv(attr(opt, "derivs"), opt$par, ctrl = control
## $checkConv, : Model failed to converge: degenerate Hessian with 1 negative
## eigenvalues
\end{verbatim}

\begin{verbatim}
## Warning in checkConv(attr(opt, "derivs"), opt$par, ctrl = control
## $checkConv, : Model failed to converge with max|grad| = 0.00142349 (tol =
## 0.001, component 1)
\end{verbatim}

\begin{verbatim}
## Warning in checkConv(attr(opt, "derivs"), opt$par, ctrl = control
## $checkConv, : Model failed to converge with max|grad| = 0.0107006 (tol =
## 0.001, component 1)
\end{verbatim}

\begin{verbatim}
## Warning in checkConv(attr(opt, "derivs"), opt$par, ctrl = control
## $checkConv, : unable to evaluate scaled gradient
\end{verbatim}

\begin{verbatim}
## Warning in checkConv(attr(opt, "derivs"), opt$par, ctrl = control
## $checkConv, : Model failed to converge: degenerate Hessian with 1 negative
## eigenvalues
\end{verbatim}

\begin{verbatim}
## Warning in checkConv(attr(opt, "derivs"), opt$par, ctrl = control
## $checkConv, : unable to evaluate scaled gradient
\end{verbatim}

\begin{verbatim}
## Warning in checkConv(attr(opt, "derivs"), opt$par, ctrl = control
## $checkConv, : Model failed to converge: degenerate Hessian with 1 negative
## eigenvalues
\end{verbatim}

\begin{verbatim}
## Warning in checkConv(attr(opt, "derivs"), opt$par, ctrl = control
## $checkConv, : Model failed to converge with max|grad| = 0.0114433 (tol =
## 0.001, component 1)
\end{verbatim}

\begin{verbatim}
## Warning in checkConv(attr(opt, "derivs"), opt$par, ctrl = control
## $checkConv, : Model failed to converge with max|grad| = 0.0165767 (tol =
## 0.001, component 1)
\end{verbatim}

\begin{verbatim}
## Warning in checkConv(attr(opt, "derivs"), opt$par, ctrl = control
## $checkConv, : Model failed to converge with max|grad| = 0.008785 (tol =
## 0.001, component 1)
\end{verbatim}

\begin{verbatim}
## Warning in checkConv(attr(opt, "derivs"), opt$par, ctrl = control
## $checkConv, : unable to evaluate scaled gradient
\end{verbatim}

\begin{verbatim}
## Warning in checkConv(attr(opt, "derivs"), opt$par, ctrl = control
## $checkConv, : Model failed to converge: degenerate Hessian with 1 negative
## eigenvalues
\end{verbatim}

\begin{verbatim}
## Warning in checkConv(attr(opt, "derivs"), opt$par, ctrl = control
## $checkConv, : Model failed to converge with max|grad| = 0.00117713 (tol =
## 0.001, component 1)
\end{verbatim}

\begin{verbatim}
## Warning in checkConv(attr(opt, "derivs"), opt$par, ctrl = control
## $checkConv, : unable to evaluate scaled gradient
\end{verbatim}

\begin{verbatim}
## Warning in checkConv(attr(opt, "derivs"), opt$par, ctrl = control
## $checkConv, : Model failed to converge: degenerate Hessian with 1 negative
## eigenvalues
\end{verbatim}

\begin{verbatim}
## Warning in checkConv(attr(opt, "derivs"), opt$par, ctrl = control
## $checkConv, : Model failed to converge with max|grad| = 0.013026 (tol =
## 0.001, component 1)
\end{verbatim}

\begin{verbatim}
## Warning in checkConv(attr(opt, "derivs"), opt$par, ctrl = control
## $checkConv, : Model failed to converge with max|grad| = 0.0125571 (tol =
## 0.001, component 1)
\end{verbatim}

\begin{verbatim}
## Warning in checkConv(attr(opt, "derivs"), opt$par, ctrl = control
## $checkConv, : unable to evaluate scaled gradient
\end{verbatim}

\begin{verbatim}
## Warning in checkConv(attr(opt, "derivs"), opt$par, ctrl = control
## $checkConv, : Model failed to converge: degenerate Hessian with 1 negative
## eigenvalues
\end{verbatim}

\begin{verbatim}
## Warning in checkConv(attr(opt, "derivs"), opt$par, ctrl = control
## $checkConv, : unable to evaluate scaled gradient
\end{verbatim}

\begin{verbatim}
## Warning in checkConv(attr(opt, "derivs"), opt$par, ctrl = control
## $checkConv, : Model failed to converge: degenerate Hessian with 1 negative
## eigenvalues
\end{verbatim}

\begin{verbatim}
## Warning in checkConv(attr(opt, "derivs"), opt$par, ctrl = control
## $checkConv, : Model failed to converge with max|grad| = 0.00749771 (tol =
## 0.001, component 1)
\end{verbatim}

\begin{verbatim}
## Warning in checkConv(attr(opt, "derivs"), opt$par, ctrl = control
## $checkConv, : unable to evaluate scaled gradient
\end{verbatim}

\begin{verbatim}
## Warning in checkConv(attr(opt, "derivs"), opt$par, ctrl = control
## $checkConv, : Model failed to converge: degenerate Hessian with 1 negative
## eigenvalues
\end{verbatim}

\begin{verbatim}
## Warning in checkConv(attr(opt, "derivs"), opt$par, ctrl = control
## $checkConv, : Model failed to converge with max|grad| = 0.00144338 (tol =
## 0.001, component 1)
\end{verbatim}

\begin{verbatim}
## Warning in checkConv(attr(opt, "derivs"), opt$par, ctrl = control
## $checkConv, : unable to evaluate scaled gradient
\end{verbatim}

\begin{verbatim}
## Warning in checkConv(attr(opt, "derivs"), opt$par, ctrl = control
## $checkConv, : Model failed to converge: degenerate Hessian with 1 negative
## eigenvalues
\end{verbatim}

\begin{verbatim}
## Warning in checkConv(attr(opt, "derivs"), opt$par, ctrl = control
## $checkConv, : unable to evaluate scaled gradient
\end{verbatim}

\begin{verbatim}
## Warning in checkConv(attr(opt, "derivs"), opt$par, ctrl = control
## $checkConv, : Model failed to converge: degenerate Hessian with 1 negative
## eigenvalues
\end{verbatim}

\begin{verbatim}
## Warning in checkConv(attr(opt, "derivs"), opt$par, ctrl = control
## $checkConv, : Model failed to converge with max|grad| = 0.00153821 (tol =
## 0.001, component 1)
\end{verbatim}

\begin{verbatim}
## Warning in checkConv(attr(opt, "derivs"), opt$par, ctrl = control
## $checkConv, : unable to evaluate scaled gradient
\end{verbatim}

\begin{verbatim}
## Warning in checkConv(attr(opt, "derivs"), opt$par, ctrl = control
## $checkConv, : Model failed to converge: degenerate Hessian with 1 negative
## eigenvalues
\end{verbatim}

\begin{verbatim}
## Warning in checkConv(attr(opt, "derivs"), opt$par, ctrl = control
## $checkConv, : unable to evaluate scaled gradient
\end{verbatim}

\begin{verbatim}
## Warning in checkConv(attr(opt, "derivs"), opt$par, ctrl = control
## $checkConv, : Model failed to converge: degenerate Hessian with 1 negative
## eigenvalues
\end{verbatim}

\begin{verbatim}
## Warning in checkConv(attr(opt, "derivs"), opt$par, ctrl = control
## $checkConv, : unable to evaluate scaled gradient
\end{verbatim}

\begin{verbatim}
## Warning in checkConv(attr(opt, "derivs"), opt$par, ctrl = control
## $checkConv, : Model failed to converge: degenerate Hessian with 1 negative
## eigenvalues
\end{verbatim}

\begin{verbatim}
## Warning in checkConv(attr(opt, "derivs"), opt$par, ctrl = control
## $checkConv, : Model failed to converge with max|grad| = 0.0020208 (tol =
## 0.001, component 1)
\end{verbatim}

\begin{verbatim}
## Warning in checkConv(attr(opt, "derivs"), opt$par, ctrl = control
## $checkConv, : Model failed to converge with max|grad| = 0.0180921 (tol =
## 0.001, component 1)
\end{verbatim}

\begin{verbatim}
## Warning in checkConv(attr(opt, "derivs"), opt$par, ctrl = control
## $checkConv, : unable to evaluate scaled gradient
\end{verbatim}

\begin{verbatim}
## Warning in checkConv(attr(opt, "derivs"), opt$par, ctrl = control
## $checkConv, : Model failed to converge: degenerate Hessian with 1 negative
## eigenvalues
\end{verbatim}

\begin{verbatim}
## Warning in checkConv(attr(opt, "derivs"), opt$par, ctrl = control
## $checkConv, : unable to evaluate scaled gradient
\end{verbatim}

\begin{verbatim}
## Warning in checkConv(attr(opt, "derivs"), opt$par, ctrl = control
## $checkConv, : Model failed to converge: degenerate Hessian with 1 negative
## eigenvalues
\end{verbatim}

\begin{verbatim}
## Warning in checkConv(attr(opt, "derivs"), opt$par, ctrl = control
## $checkConv, : Model failed to converge with max|grad| = 0.024572 (tol =
## 0.001, component 1)
\end{verbatim}

\begin{verbatim}
## Warning in checkConv(attr(opt, "derivs"), opt$par, ctrl = control
## $checkConv, : unable to evaluate scaled gradient
\end{verbatim}

\begin{verbatim}
## Warning in checkConv(attr(opt, "derivs"), opt$par, ctrl = control
## $checkConv, : Model failed to converge: degenerate Hessian with 1 negative
## eigenvalues
\end{verbatim}

\begin{verbatim}
## Warning in checkConv(attr(opt, "derivs"), opt$par, ctrl = control
## $checkConv, : Model failed to converge with max|grad| = 0.0155976 (tol =
## 0.001, component 1)
\end{verbatim}

\begin{verbatim}
## Warning in checkConv(attr(opt, "derivs"), opt$par, ctrl = control
## $checkConv, : unable to evaluate scaled gradient
\end{verbatim}

\begin{verbatim}
## Warning in checkConv(attr(opt, "derivs"), opt$par, ctrl = control
## $checkConv, : Model failed to converge: degenerate Hessian with 1 negative
## eigenvalues
\end{verbatim}

\begin{Shaded}
\begin{Highlighting}[]
\NormalTok{bootconf}
\end{Highlighting}
\end{Shaded}

\begin{verbatim}
##                      2.5 %     97.5 %
## .sig01        3.390255e-06  0.8101804
## (Intercept)  -2.085402e+00 -0.6366033
## daycent       4.147421e-02  0.2778558
## SEXM         -1.245648e+00  0.8446987
## daycent:SEXM -1.304753e-01  0.2070392
\end{verbatim}

\subsubsection{This comes to a very similar, unfortunately ambiguous
result.}\label{this-comes-to-a-very-similar-unfortunately-ambiguous-result.}

\subsection{Regardless, there is strong evidence that there is a change
in the success rate through time and then slope is positive, leading to
the conclusion that frogs are getting better in time. The interaction
term would suggest, if significant, that males learn more slowly than
females.}\label{regardless-there-is-strong-evidence-that-there-is-a-change-in-the-success-rate-through-time-and-then-slope-is-positive-leading-to-the-conclusion-that-frogs-are-getting-better-in-time.-the-interaction-term-would-suggest-if-significant-that-males-learn-more-slowly-than-females.}

\section{We can do the same thing for NC\_errors as
well\ldots{}}\label{we-can-do-the-same-thing-for-nc_errors-as-well}

\begin{Shaded}
\begin{Highlighting}[]
\KeywordTok{set.seed}\NormalTok{(}\DecValTok{4}\NormalTok{)}
\NormalTok{bootconf <-}\StringTok{ }\KeywordTok{confint}\NormalTok{(NC1glmmfull,}\DataTypeTok{method=}\StringTok{"boot"}\NormalTok{,}\DataTypeTok{nsim=}\DecValTok{2000}\NormalTok{)}
\end{Highlighting}
\end{Shaded}

\begin{verbatim}
## Computing bootstrap confidence intervals ...
\end{verbatim}

\begin{verbatim}
## Warning in checkConv(attr(opt, "derivs"), opt$par, ctrl = control
## $checkConv, : Model failed to converge with max|grad| = 0.128954 (tol =
## 0.001, component 1)
\end{verbatim}

\begin{verbatim}
## Warning in checkConv(attr(opt, "derivs"), opt$par, ctrl = control
## $checkConv, : Model failed to converge with max|grad| = 0.00315928 (tol =
## 0.001, component 1)
\end{verbatim}

\begin{verbatim}
## Warning in checkConv(attr(opt, "derivs"), opt$par, ctrl = control
## $checkConv, : Model failed to converge with max|grad| = 0.0231425 (tol =
## 0.001, component 1)
\end{verbatim}

\begin{verbatim}
## Warning in checkConv(attr(opt, "derivs"), opt$par, ctrl = control
## $checkConv, : Model failed to converge with max|grad| = 0.00124692 (tol =
## 0.001, component 1)
\end{verbatim}

\begin{verbatim}
## Warning in checkConv(attr(opt, "derivs"), opt$par, ctrl = control
## $checkConv, : Model failed to converge with max|grad| = 0.0258766 (tol =
## 0.001, component 1)
\end{verbatim}

\begin{verbatim}
## Warning in checkConv(attr(opt, "derivs"), opt$par, ctrl = control
## $checkConv, : Model failed to converge with max|grad| = 0.0190982 (tol =
## 0.001, component 1)
\end{verbatim}

\begin{verbatim}
## Warning in checkConv(attr(opt, "derivs"), opt$par, ctrl = control
## $checkConv, : Model failed to converge with max|grad| = 0.00108052 (tol =
## 0.001, component 1)
\end{verbatim}

\begin{verbatim}
## Warning in checkConv(attr(opt, "derivs"), opt$par, ctrl = control
## $checkConv, : Model failed to converge with max|grad| = 0.0212206 (tol =
## 0.001, component 1)
\end{verbatim}

\begin{verbatim}
## Warning in checkConv(attr(opt, "derivs"), opt$par, ctrl = control
## $checkConv, : unable to evaluate scaled gradient
\end{verbatim}

\begin{verbatim}
## Warning in checkConv(attr(opt, "derivs"), opt$par, ctrl = control
## $checkConv, : Model failed to converge: degenerate Hessian with 1 negative
## eigenvalues
\end{verbatim}

\begin{Shaded}
\begin{Highlighting}[]
\NormalTok{bootconf}
\end{Highlighting}
\end{Shaded}

\begin{verbatim}
##                    2.5 %     97.5 %
## .sig01        0.44510732  1.0932722
## (Intercept)  -1.08089406  0.1285544
## daycent      -0.34791822 -0.1449100
## SEXM         -1.13037678  0.5861909
## daycent:SEXM -0.03924377  0.2278232
\end{verbatim}

For experiment 1, it seems that there is strong evidence that the
decrease in NC\_errors is significant and negative, based on the model
ANOVA and these bootstrap confidence intervals.

\begin{Shaded}
\begin{Highlighting}[]
\KeywordTok{set.seed}\NormalTok{(}\DecValTok{4}\NormalTok{)}
\NormalTok{bootconf <-}\StringTok{ }\KeywordTok{confint}\NormalTok{(NC2glmmfull,}\DataTypeTok{method=}\StringTok{"boot"}\NormalTok{,}\DataTypeTok{nsim=}\DecValTok{2000}\NormalTok{)}
\end{Highlighting}
\end{Shaded}

\begin{verbatim}
## Computing bootstrap confidence intervals ...
\end{verbatim}

\begin{verbatim}
## Warning in checkConv(attr(opt, "derivs"), opt$par, ctrl = control
## $checkConv, : Model failed to converge with max|grad| = 0.0247132 (tol =
## 0.001, component 1)
\end{verbatim}

\begin{verbatim}
## Warning in checkConv(attr(opt, "derivs"), opt$par, ctrl = control
## $checkConv, : Model failed to converge with max|grad| = 0.0251921 (tol =
## 0.001, component 1)
\end{verbatim}

\begin{verbatim}
## Warning in checkConv(attr(opt, "derivs"), opt$par, ctrl = control
## $checkConv, : Model failed to converge with max|grad| = 0.0215523 (tol =
## 0.001, component 1)
\end{verbatim}

\begin{verbatim}
## Warning in checkConv(attr(opt, "derivs"), opt$par, ctrl = control
## $checkConv, : Model failed to converge with max|grad| = 0.0243809 (tol =
## 0.001, component 1)
\end{verbatim}

\begin{verbatim}
## Warning in checkConv(attr(opt, "derivs"), opt$par, ctrl = control
## $checkConv, : unable to evaluate scaled gradient
\end{verbatim}

\begin{verbatim}
## Warning in checkConv(attr(opt, "derivs"), opt$par, ctrl = control
## $checkConv, : Model failed to converge: degenerate Hessian with 1 negative
## eigenvalues
\end{verbatim}

\begin{verbatim}
## Warning in checkConv(attr(opt, "derivs"), opt$par, ctrl = control
## $checkConv, : Model failed to converge with max|grad| = 0.0203674 (tol =
## 0.001, component 1)
\end{verbatim}

\begin{verbatim}
## Warning in checkConv(attr(opt, "derivs"), opt$par, ctrl = control
## $checkConv, : Model failed to converge with max|grad| = 0.0144993 (tol =
## 0.001, component 1)
\end{verbatim}

\begin{verbatim}
## Warning in checkConv(attr(opt, "derivs"), opt$par, ctrl = control
## $checkConv, : Model failed to converge with max|grad| = 0.0250504 (tol =
## 0.001, component 1)
\end{verbatim}

\begin{verbatim}
## Warning in checkConv(attr(opt, "derivs"), opt$par, ctrl = control
## $checkConv, : Model failed to converge with max|grad| = 0.0228481 (tol =
## 0.001, component 1)
\end{verbatim}

\begin{verbatim}
## Warning in checkConv(attr(opt, "derivs"), opt$par, ctrl = control
## $checkConv, : Model failed to converge with max|grad| = 0.0202427 (tol =
## 0.001, component 1)
\end{verbatim}

\begin{verbatim}
## Warning in checkConv(attr(opt, "derivs"), opt$par, ctrl = control
## $checkConv, : Model failed to converge with max|grad| = 0.00145026 (tol =
## 0.001, component 1)
\end{verbatim}

\begin{verbatim}
## Warning in checkConv(attr(opt, "derivs"), opt$par, ctrl = control
## $checkConv, : Model failed to converge with max|grad| = 0.0184132 (tol =
## 0.001, component 1)
\end{verbatim}

\begin{verbatim}
## Warning in checkConv(attr(opt, "derivs"), opt$par, ctrl = control
## $checkConv, : Model failed to converge with max|grad| = 0.025683 (tol =
## 0.001, component 1)
\end{verbatim}

\begin{verbatim}
## Warning in checkConv(attr(opt, "derivs"), opt$par, ctrl = control
## $checkConv, : Model failed to converge with max|grad| = 0.0199906 (tol =
## 0.001, component 1)
\end{verbatim}

\begin{verbatim}
## Warning in checkConv(attr(opt, "derivs"), opt$par, ctrl = control
## $checkConv, : Model failed to converge with max|grad| = 0.0238521 (tol =
## 0.001, component 1)
\end{verbatim}

\begin{verbatim}
## Warning in checkConv(attr(opt, "derivs"), opt$par, ctrl = control
## $checkConv, : Model failed to converge with max|grad| = 0.02372 (tol =
## 0.001, component 1)
\end{verbatim}

\begin{verbatim}
## Warning in checkConv(attr(opt, "derivs"), opt$par, ctrl = control
## $checkConv, : Model failed to converge with max|grad| = 0.0220664 (tol =
## 0.001, component 1)
\end{verbatim}

\begin{verbatim}
## Warning in checkConv(attr(opt, "derivs"), opt$par, ctrl = control
## $checkConv, : Model failed to converge with max|grad| = 0.0200732 (tol =
## 0.001, component 1)
\end{verbatim}

\begin{verbatim}
## Warning in checkConv(attr(opt, "derivs"), opt$par, ctrl = control
## $checkConv, : Model failed to converge with max|grad| = 0.0221137 (tol =
## 0.001, component 1)
\end{verbatim}

\begin{verbatim}
## Warning in checkConv(attr(opt, "derivs"), opt$par, ctrl = control
## $checkConv, : Model failed to converge with max|grad| = 0.0211275 (tol =
## 0.001, component 1)
\end{verbatim}

\begin{verbatim}
## Warning in checkConv(attr(opt, "derivs"), opt$par, ctrl = control
## $checkConv, : Model failed to converge with max|grad| = 0.0186918 (tol =
## 0.001, component 1)
\end{verbatim}

\begin{verbatim}
## Warning in checkConv(attr(opt, "derivs"), opt$par, ctrl = control
## $checkConv, : Model failed to converge with max|grad| = 0.0245802 (tol =
## 0.001, component 1)
\end{verbatim}

\begin{verbatim}
## Warning in checkConv(attr(opt, "derivs"), opt$par, ctrl = control
## $checkConv, : Model failed to converge with max|grad| = 0.0243605 (tol =
## 0.001, component 1)
\end{verbatim}

\begin{verbatim}
## Warning in checkConv(attr(opt, "derivs"), opt$par, ctrl = control
## $checkConv, : Model failed to converge with max|grad| = 0.0224123 (tol =
## 0.001, component 1)
\end{verbatim}

\begin{verbatim}
## Warning in checkConv(attr(opt, "derivs"), opt$par, ctrl = control
## $checkConv, : Model failed to converge with max|grad| = 0.0174547 (tol =
## 0.001, component 1)
\end{verbatim}

\begin{verbatim}
## Warning in checkConv(attr(opt, "derivs"), opt$par, ctrl = control
## $checkConv, : unable to evaluate scaled gradient
\end{verbatim}

\begin{verbatim}
## Warning in checkConv(attr(opt, "derivs"), opt$par, ctrl = control
## $checkConv, : Model failed to converge: degenerate Hessian with 1 negative
## eigenvalues
\end{verbatim}

\begin{verbatim}
## Warning in checkConv(attr(opt, "derivs"), opt$par, ctrl = control
## $checkConv, : Model failed to converge with max|grad| = 0.0275114 (tol =
## 0.001, component 1)
\end{verbatim}

\begin{verbatim}
## Warning in checkConv(attr(opt, "derivs"), opt$par, ctrl = control
## $checkConv, : Model failed to converge with max|grad| = 0.0195503 (tol =
## 0.001, component 1)
\end{verbatim}

\begin{Shaded}
\begin{Highlighting}[]
\NormalTok{bootconf}
\end{Highlighting}
\end{Shaded}

\begin{verbatim}
##                   2.5 %     97.5 %
## .sig01        0.3362997  1.6725883
## (Intercept)  -1.9631716  0.3125134
## daycent      -0.7442877 -0.2777528
## SEXM         -1.2944339  1.8993449
## daycent:SEXM -0.1696709  0.4232826
\end{verbatim}

A similar result is seen for experiment 2 NC\_errors - the slope is
negative and significant, and evidence suggest frogs are making fewer
NC\_errors over time.

\section{P\_error bootstrapping\ldots{}}\label{p_error-bootstrapping}

\begin{Shaded}
\begin{Highlighting}[]
\KeywordTok{set.seed}\NormalTok{(}\DecValTok{4}\NormalTok{)}
\NormalTok{bootconf <-}\StringTok{ }\KeywordTok{confint}\NormalTok{(P1glmmfull,}\DataTypeTok{method=}\StringTok{"boot"}\NormalTok{,}\DataTypeTok{nsim=}\DecValTok{2000}\NormalTok{)}
\end{Highlighting}
\end{Shaded}

\begin{verbatim}
## Computing bootstrap confidence intervals ...
\end{verbatim}

\begin{verbatim}
## Warning in checkConv(attr(opt, "derivs"), opt$par, ctrl = control
## $checkConv, : Model failed to converge with max|grad| = 0.0018014 (tol =
## 0.001, component 1)
\end{verbatim}

\begin{verbatim}
## Warning in checkConv(attr(opt, "derivs"), opt$par, ctrl = control
## $checkConv, : Model failed to converge with max|grad| = 0.029158 (tol =
## 0.001, component 1)
\end{verbatim}

\begin{verbatim}
## Warning in checkConv(attr(opt, "derivs"), opt$par, ctrl = control
## $checkConv, : Model failed to converge with max|grad| = 0.0271536 (tol =
## 0.001, component 1)
\end{verbatim}

\begin{verbatim}
## Warning in checkConv(attr(opt, "derivs"), opt$par, ctrl = control
## $checkConv, : unable to evaluate scaled gradient
\end{verbatim}

\begin{verbatim}
## Warning in checkConv(attr(opt, "derivs"), opt$par, ctrl = control
## $checkConv, : Model failed to converge: degenerate Hessian with 1 negative
## eigenvalues
\end{verbatim}

\begin{verbatim}
## Warning in checkConv(attr(opt, "derivs"), opt$par, ctrl = control
## $checkConv, : Model failed to converge with max|grad| = 0.0254617 (tol =
## 0.001, component 1)
\end{verbatim}

\begin{verbatim}
## Warning in checkConv(attr(opt, "derivs"), opt$par, ctrl = control
## $checkConv, : Model failed to converge with max|grad| = 0.0254106 (tol =
## 0.001, component 1)
\end{verbatim}

\begin{verbatim}
## Warning in checkConv(attr(opt, "derivs"), opt$par, ctrl = control
## $checkConv, : Model failed to converge with max|grad| = 0.0212181 (tol =
## 0.001, component 1)
\end{verbatim}

\begin{verbatim}
## Warning in checkConv(attr(opt, "derivs"), opt$par, ctrl = control
## $checkConv, : Model failed to converge with max|grad| = 0.0300388 (tol =
## 0.001, component 1)
\end{verbatim}

\begin{verbatim}
## Warning in checkConv(attr(opt, "derivs"), opt$par, ctrl = control
## $checkConv, : Model failed to converge with max|grad| = 0.024085 (tol =
## 0.001, component 1)
\end{verbatim}

\begin{verbatim}
## Warning in checkConv(attr(opt, "derivs"), opt$par, ctrl = control
## $checkConv, : Model failed to converge with max|grad| = 0.0324246 (tol =
## 0.001, component 1)
\end{verbatim}

\begin{verbatim}
## Warning in checkConv(attr(opt, "derivs"), opt$par, ctrl = control
## $checkConv, : Model failed to converge with max|grad| = 0.0250077 (tol =
## 0.001, component 1)
\end{verbatim}

\begin{verbatim}
## Warning in checkConv(attr(opt, "derivs"), opt$par, ctrl = control
## $checkConv, : Model failed to converge with max|grad| = 0.0229923 (tol =
## 0.001, component 1)
\end{verbatim}

\begin{verbatim}
## Warning in checkConv(attr(opt, "derivs"), opt$par, ctrl = control
## $checkConv, : unable to evaluate scaled gradient
\end{verbatim}

\begin{verbatim}
## Warning in checkConv(attr(opt, "derivs"), opt$par, ctrl = control
## $checkConv, : Model failed to converge: degenerate Hessian with 1 negative
## eigenvalues
\end{verbatim}

\begin{verbatim}
## Warning in checkConv(attr(opt, "derivs"), opt$par, ctrl = control
## $checkConv, : Model failed to converge with max|grad| = 0.0270156 (tol =
## 0.001, component 1)
\end{verbatim}

\begin{verbatim}
## Warning in checkConv(attr(opt, "derivs"), opt$par, ctrl = control
## $checkConv, : Model failed to converge with max|grad| = 0.00147996 (tol =
## 0.001, component 1)
\end{verbatim}

\begin{verbatim}
## Warning in checkConv(attr(opt, "derivs"), opt$par, ctrl = control
## $checkConv, : Model failed to converge with max|grad| = 0.00102749 (tol =
## 0.001, component 1)
\end{verbatim}

\begin{verbatim}
## Warning in checkConv(attr(opt, "derivs"), opt$par, ctrl = control
## $checkConv, : Model failed to converge with max|grad| = 0.00261247 (tol =
## 0.001, component 1)
\end{verbatim}

\begin{verbatim}
## Warning in checkConv(attr(opt, "derivs"), opt$par, ctrl = control
## $checkConv, : Model failed to converge with max|grad| = 0.0269778 (tol =
## 0.001, component 1)
\end{verbatim}

\begin{verbatim}
## Warning in checkConv(attr(opt, "derivs"), opt$par, ctrl = control
## $checkConv, : Model failed to converge with max|grad| = 0.00125751 (tol =
## 0.001, component 1)
\end{verbatim}

\begin{verbatim}
## Warning in checkConv(attr(opt, "derivs"), opt$par, ctrl = control
## $checkConv, : Model failed to converge with max|grad| = 0.0256878 (tol =
## 0.001, component 1)
\end{verbatim}

\begin{verbatim}
## Warning in checkConv(attr(opt, "derivs"), opt$par, ctrl = control
## $checkConv, : Model failed to converge with max|grad| = 0.00131793 (tol =
## 0.001, component 1)
\end{verbatim}

\begin{verbatim}
## Warning in checkConv(attr(opt, "derivs"), opt$par, ctrl = control
## $checkConv, : Model failed to converge with max|grad| = 0.00101968 (tol =
## 0.001, component 1)
\end{verbatim}

\begin{verbatim}
## Warning in checkConv(attr(opt, "derivs"), opt$par, ctrl = control
## $checkConv, : Model failed to converge with max|grad| = 0.0217811 (tol =
## 0.001, component 1)
\end{verbatim}

\begin{verbatim}
## Warning in checkConv(attr(opt, "derivs"), opt$par, ctrl = control
## $checkConv, : Model failed to converge with max|grad| = 0.00133438 (tol =
## 0.001, component 1)
\end{verbatim}

\begin{verbatim}
## Warning in checkConv(attr(opt, "derivs"), opt$par, ctrl = control
## $checkConv, : Model failed to converge with max|grad| = 0.0288367 (tol =
## 0.001, component 1)
\end{verbatim}

\begin{verbatim}
## Warning in checkConv(attr(opt, "derivs"), opt$par, ctrl = control
## $checkConv, : Model failed to converge with max|grad| = 0.0246017 (tol =
## 0.001, component 1)
\end{verbatim}

\begin{verbatim}
## Warning in checkConv(attr(opt, "derivs"), opt$par, ctrl = control
## $checkConv, : unable to evaluate scaled gradient
\end{verbatim}

\begin{verbatim}
## Warning in checkConv(attr(opt, "derivs"), opt$par, ctrl = control
## $checkConv, : Model failed to converge: degenerate Hessian with 1 negative
## eigenvalues
\end{verbatim}

\begin{verbatim}
## Warning in checkConv(attr(opt, "derivs"), opt$par, ctrl = control
## $checkConv, : unable to evaluate scaled gradient
\end{verbatim}

\begin{verbatim}
## Warning in checkConv(attr(opt, "derivs"), opt$par, ctrl = control
## $checkConv, : Model failed to converge: degenerate Hessian with 1 negative
## eigenvalues
\end{verbatim}

\begin{verbatim}
## Warning in checkConv(attr(opt, "derivs"), opt$par, ctrl = control
## $checkConv, : unable to evaluate scaled gradient
\end{verbatim}

\begin{verbatim}
## Warning in checkConv(attr(opt, "derivs"), opt$par, ctrl = control
## $checkConv, : Model failed to converge: degenerate Hessian with 1 negative
## eigenvalues
\end{verbatim}

\begin{verbatim}
## Warning in checkConv(attr(opt, "derivs"), opt$par, ctrl = control
## $checkConv, : Model failed to converge with max|grad| = 0.00105756 (tol =
## 0.001, component 1)
\end{verbatim}

\begin{verbatim}
## Warning in checkConv(attr(opt, "derivs"), opt$par, ctrl = control
## $checkConv, : Model failed to converge with max|grad| = 0.0278278 (tol =
## 0.001, component 1)
\end{verbatim}

\begin{verbatim}
## Warning in checkConv(attr(opt, "derivs"), opt$par, ctrl = control
## $checkConv, : Model failed to converge with max|grad| = 0.00119803 (tol =
## 0.001, component 1)
\end{verbatim}

\begin{verbatim}
## Warning in checkConv(attr(opt, "derivs"), opt$par, ctrl = control
## $checkConv, : Model failed to converge with max|grad| = 0.0285873 (tol =
## 0.001, component 1)
\end{verbatim}

\begin{verbatim}
## Warning in checkConv(attr(opt, "derivs"), opt$par, ctrl = control
## $checkConv, : Model failed to converge with max|grad| = 0.0230871 (tol =
## 0.001, component 1)
\end{verbatim}

\begin{verbatim}
## Warning in checkConv(attr(opt, "derivs"), opt$par, ctrl = control
## $checkConv, : Model failed to converge with max|grad| = 0.0236013 (tol =
## 0.001, component 1)
\end{verbatim}

\begin{verbatim}
## Warning in checkConv(attr(opt, "derivs"), opt$par, ctrl = control
## $checkConv, : Model failed to converge with max|grad| = 0.0277011 (tol =
## 0.001, component 1)
\end{verbatim}

\begin{verbatim}
## Warning in checkConv(attr(opt, "derivs"), opt$par, ctrl = control
## $checkConv, : Model failed to converge with max|grad| = 0.00107735 (tol =
## 0.001, component 1)
\end{verbatim}

\begin{verbatim}
## Warning in checkConv(attr(opt, "derivs"), opt$par, ctrl = control
## $checkConv, : unable to evaluate scaled gradient
\end{verbatim}

\begin{verbatim}
## Warning in checkConv(attr(opt, "derivs"), opt$par, ctrl = control
## $checkConv, : Model failed to converge: degenerate Hessian with 1 negative
## eigenvalues
\end{verbatim}

\begin{verbatim}
## Warning in checkConv(attr(opt, "derivs"), opt$par, ctrl = control
## $checkConv, : Model failed to converge with max|grad| = 0.0274261 (tol =
## 0.001, component 1)
\end{verbatim}

\begin{verbatim}
## Warning in checkConv(attr(opt, "derivs"), opt$par, ctrl = control
## $checkConv, : Model failed to converge with max|grad| = 0.0233047 (tol =
## 0.001, component 1)
\end{verbatim}

\begin{verbatim}
## Warning in checkConv(attr(opt, "derivs"), opt$par, ctrl = control
## $checkConv, : Model failed to converge with max|grad| = 0.00113963 (tol =
## 0.001, component 1)
\end{verbatim}

\begin{verbatim}
## Warning in checkConv(attr(opt, "derivs"), opt$par, ctrl = control
## $checkConv, : Model failed to converge with max|grad| = 0.00100417 (tol =
## 0.001, component 1)
\end{verbatim}

\begin{verbatim}
## Warning in checkConv(attr(opt, "derivs"), opt$par, ctrl = control
## $checkConv, : Model failed to converge with max|grad| = 0.0257858 (tol =
## 0.001, component 1)
\end{verbatim}

\begin{verbatim}
## Warning in checkConv(attr(opt, "derivs"), opt$par, ctrl = control
## $checkConv, : Model failed to converge with max|grad| = 0.0261353 (tol =
## 0.001, component 1)
\end{verbatim}

\begin{verbatim}
## Warning in checkConv(attr(opt, "derivs"), opt$par, ctrl = control
## $checkConv, : Model failed to converge with max|grad| = 0.0236599 (tol =
## 0.001, component 1)
\end{verbatim}

\begin{verbatim}
## Warning in checkConv(attr(opt, "derivs"), opt$par, ctrl = control
## $checkConv, : Model failed to converge with max|grad| = 0.00129139 (tol =
## 0.001, component 1)
\end{verbatim}

\begin{verbatim}
## Warning in checkConv(attr(opt, "derivs"), opt$par, ctrl = control
## $checkConv, : Model failed to converge with max|grad| = 0.0246024 (tol =
## 0.001, component 1)
\end{verbatim}

\begin{verbatim}
## Warning in checkConv(attr(opt, "derivs"), opt$par, ctrl = control
## $checkConv, : Model failed to converge with max|grad| = 0.0294133 (tol =
## 0.001, component 1)
\end{verbatim}

\begin{verbatim}
## Warning in checkConv(attr(opt, "derivs"), opt$par, ctrl = control
## $checkConv, : Model failed to converge with max|grad| = 0.0248965 (tol =
## 0.001, component 1)
\end{verbatim}

\begin{verbatim}
## Warning in checkConv(attr(opt, "derivs"), opt$par, ctrl = control
## $checkConv, : Model failed to converge with max|grad| = 0.0193529 (tol =
## 0.001, component 1)
\end{verbatim}

\begin{verbatim}
## Warning in checkConv(attr(opt, "derivs"), opt$par, ctrl = control
## $checkConv, : Model failed to converge with max|grad| = 0.02138 (tol =
## 0.001, component 1)
\end{verbatim}

\begin{verbatim}
## Warning in checkConv(attr(opt, "derivs"), opt$par, ctrl = control
## $checkConv, : Model failed to converge with max|grad| = 0.0013785 (tol =
## 0.001, component 1)
\end{verbatim}

\begin{verbatim}
## Warning in checkConv(attr(opt, "derivs"), opt$par, ctrl = control
## $checkConv, : Model failed to converge with max|grad| = 0.0252757 (tol =
## 0.001, component 1)
\end{verbatim}

\begin{verbatim}
## Warning in checkConv(attr(opt, "derivs"), opt$par, ctrl = control
## $checkConv, : Model failed to converge with max|grad| = 0.00348144 (tol =
## 0.001, component 1)
\end{verbatim}

\begin{Shaded}
\begin{Highlighting}[]
\NormalTok{bootconf}
\end{Highlighting}
\end{Shaded}

\begin{verbatim}
##                    2.5 %     97.5 %
## .sig01        0.14716685 0.46786852
## (Intercept)  -0.02244658 0.60963713
## daycent       0.01793050 0.11124946
## SEXM         -0.53695221 0.33772882
## daycent:SEXM -0.08332505 0.05812018
\end{verbatim}

Unfortunately this result doesn't seem to reveal anything significant -
although P\_errors do vary significantly by day as determined by the
ANOVA above, evidence suggests that the slope is positive.

\begin{Shaded}
\begin{Highlighting}[]
\KeywordTok{set.seed}\NormalTok{(}\DecValTok{4}\NormalTok{)}
\NormalTok{bootconf <-}\StringTok{ }\KeywordTok{confint}\NormalTok{(P2glmmfull,}\DataTypeTok{method=}\StringTok{"boot"}\NormalTok{,}\DataTypeTok{nsim=}\DecValTok{2000}\NormalTok{)}
\end{Highlighting}
\end{Shaded}

\begin{verbatim}
## Computing bootstrap confidence intervals ...
\end{verbatim}

\begin{verbatim}
## Warning in checkConv(attr(opt, "derivs"), opt$par, ctrl = control
## $checkConv, : Model failed to converge with max|grad| = 0.15655 (tol =
## 0.001, component 1)
\end{verbatim}

\begin{verbatim}
## Warning in checkConv(attr(opt, "derivs"), opt$par, ctrl = control
## $checkConv, : Model failed to converge with max|grad| = 0.0185092 (tol =
## 0.001, component 1)
\end{verbatim}

\begin{verbatim}
## Warning in checkConv(attr(opt, "derivs"), opt$par, ctrl = control
## $checkConv, : Model failed to converge with max|grad| = 0.0182919 (tol =
## 0.001, component 1)
\end{verbatim}

\begin{verbatim}
## Warning in checkConv(attr(opt, "derivs"), opt$par, ctrl = control
## $checkConv, : Model failed to converge with max|grad| = 0.0201293 (tol =
## 0.001, component 1)
\end{verbatim}

\begin{verbatim}
## Warning in checkConv(attr(opt, "derivs"), opt$par, ctrl = control
## $checkConv, : Model failed to converge with max|grad| = 0.00353477 (tol =
## 0.001, component 1)
\end{verbatim}

\begin{verbatim}
## Warning in checkConv(attr(opt, "derivs"), opt$par, ctrl = control
## $checkConv, : Model failed to converge with max|grad| = 0.00215129 (tol =
## 0.001, component 1)
\end{verbatim}

\begin{verbatim}
## Warning in checkConv(attr(opt, "derivs"), opt$par, ctrl = control
## $checkConv, : Model failed to converge with max|grad| = 0.00272206 (tol =
## 0.001, component 1)
\end{verbatim}

\begin{verbatim}
## Warning in checkConv(attr(opt, "derivs"), opt$par, ctrl = control
## $checkConv, : unable to evaluate scaled gradient
\end{verbatim}

\begin{verbatim}
## Warning in checkConv(attr(opt, "derivs"), opt$par, ctrl = control
## $checkConv, : Model failed to converge: degenerate Hessian with 1 negative
## eigenvalues
\end{verbatim}

\begin{verbatim}
## Warning in checkConv(attr(opt, "derivs"), opt$par, ctrl = control
## $checkConv, : Model failed to converge with max|grad| = 0.0185329 (tol =
## 0.001, component 1)
\end{verbatim}

\begin{verbatim}
## Warning in checkConv(attr(opt, "derivs"), opt$par, ctrl = control
## $checkConv, : Model failed to converge with max|grad| = 0.0184573 (tol =
## 0.001, component 1)
\end{verbatim}

\begin{verbatim}
## Warning in checkConv(attr(opt, "derivs"), opt$par, ctrl = control
## $checkConv, : Model failed to converge with max|grad| = 0.00232485 (tol =
## 0.001, component 1)
\end{verbatim}

\begin{verbatim}
## Warning in checkConv(attr(opt, "derivs"), opt$par, ctrl = control
## $checkConv, : Model failed to converge with max|grad| = 0.00114098 (tol =
## 0.001, component 1)
\end{verbatim}

\begin{verbatim}
## Warning in checkConv(attr(opt, "derivs"), opt$par, ctrl = control
## $checkConv, : Model failed to converge with max|grad| = 0.0192998 (tol =
## 0.001, component 1)
\end{verbatim}

\begin{verbatim}
## Warning in checkConv(attr(opt, "derivs"), opt$par, ctrl = control
## $checkConv, : Model failed to converge with max|grad| = 0.00170539 (tol =
## 0.001, component 1)
\end{verbatim}

\begin{verbatim}
## Warning in checkConv(attr(opt, "derivs"), opt$par, ctrl = control
## $checkConv, : Model failed to converge with max|grad| = 0.00281854 (tol =
## 0.001, component 1)
\end{verbatim}

\begin{verbatim}
## Warning in checkConv(attr(opt, "derivs"), opt$par, ctrl = control
## $checkConv, : Model failed to converge with max|grad| = 0.0475374 (tol =
## 0.001, component 1)
\end{verbatim}

\begin{verbatim}
## Warning in checkConv(attr(opt, "derivs"), opt$par, ctrl = control
## $checkConv, : Model failed to converge with max|grad| = 0.00209823 (tol =
## 0.001, component 1)
\end{verbatim}

\begin{verbatim}
## Warning in checkConv(attr(opt, "derivs"), opt$par, ctrl = control
## $checkConv, : Model failed to converge with max|grad| = 0.0181403 (tol =
## 0.001, component 1)
\end{verbatim}

\begin{verbatim}
## Warning in checkConv(attr(opt, "derivs"), opt$par, ctrl = control
## $checkConv, : Model failed to converge with max|grad| = 0.0178864 (tol =
## 0.001, component 1)
\end{verbatim}

\begin{verbatim}
## Warning in checkConv(attr(opt, "derivs"), opt$par, ctrl = control
## $checkConv, : Model failed to converge with max|grad| = 0.00383177 (tol =
## 0.001, component 1)
\end{verbatim}

\begin{verbatim}
## Warning in checkConv(attr(opt, "derivs"), opt$par, ctrl = control
## $checkConv, : Model failed to converge with max|grad| = 0.00187243 (tol =
## 0.001, component 1)
\end{verbatim}

\begin{verbatim}
## Warning in checkConv(attr(opt, "derivs"), opt$par, ctrl = control
## $checkConv, : Model failed to converge with max|grad| = 0.017746 (tol =
## 0.001, component 1)
\end{verbatim}

\begin{verbatim}
## Warning in checkConv(attr(opt, "derivs"), opt$par, ctrl = control
## $checkConv, : Model failed to converge with max|grad| = 0.018836 (tol =
## 0.001, component 1)
\end{verbatim}

\begin{verbatim}
## Warning in checkConv(attr(opt, "derivs"), opt$par, ctrl = control
## $checkConv, : Model failed to converge with max|grad| = 0.0186213 (tol =
## 0.001, component 1)
\end{verbatim}

\begin{verbatim}
## Warning in checkConv(attr(opt, "derivs"), opt$par, ctrl = control
## $checkConv, : Model failed to converge with max|grad| = 0.0150205 (tol =
## 0.001, component 1)
\end{verbatim}

\begin{verbatim}
## Warning in checkConv(attr(opt, "derivs"), opt$par, ctrl = control
## $checkConv, : unable to evaluate scaled gradient
\end{verbatim}

\begin{verbatim}
## Warning in checkConv(attr(opt, "derivs"), opt$par, ctrl = control
## $checkConv, : Model failed to converge: degenerate Hessian with 1 negative
## eigenvalues
\end{verbatim}

\begin{verbatim}
## Warning in checkConv(attr(opt, "derivs"), opt$par, ctrl = control
## $checkConv, : Model failed to converge with max|grad| = 0.0136483 (tol =
## 0.001, component 1)
\end{verbatim}

\begin{verbatim}
## Warning in checkConv(attr(opt, "derivs"), opt$par, ctrl = control
## $checkConv, : Model failed to converge with max|grad| = 0.00156727 (tol =
## 0.001, component 1)
\end{verbatim}

\begin{verbatim}
## Warning in checkConv(attr(opt, "derivs"), opt$par, ctrl = control
## $checkConv, : Model failed to converge with max|grad| = 0.0145968 (tol =
## 0.001, component 1)
\end{verbatim}

\begin{verbatim}
## Warning in checkConv(attr(opt, "derivs"), opt$par, ctrl = control
## $checkConv, : unable to evaluate scaled gradient
\end{verbatim}

\begin{verbatim}
## Warning in checkConv(attr(opt, "derivs"), opt$par, ctrl = control
## $checkConv, : Model failed to converge: degenerate Hessian with 1 negative
## eigenvalues
\end{verbatim}

\begin{verbatim}
## Warning in checkConv(attr(opt, "derivs"), opt$par, ctrl = control
## $checkConv, : Model failed to converge with max|grad| = 0.00245614 (tol =
## 0.001, component 1)
\end{verbatim}

\begin{verbatim}
## Warning in checkConv(attr(opt, "derivs"), opt$par, ctrl = control
## $checkConv, : Model failed to converge with max|grad| = 0.0212931 (tol =
## 0.001, component 1)
\end{verbatim}

\begin{verbatim}
## Warning in checkConv(attr(opt, "derivs"), opt$par, ctrl = control
## $checkConv, : Model failed to converge with max|grad| = 0.0096323 (tol =
## 0.001, component 1)
\end{verbatim}

\begin{verbatim}
## Warning in checkConv(attr(opt, "derivs"), opt$par, ctrl = control
## $checkConv, : unable to evaluate scaled gradient
\end{verbatim}

\begin{verbatim}
## Warning in checkConv(attr(opt, "derivs"), opt$par, ctrl = control
## $checkConv, : Model failed to converge: degenerate Hessian with 1 negative
## eigenvalues
\end{verbatim}

\begin{verbatim}
## Warning in checkConv(attr(opt, "derivs"), opt$par, ctrl = control
## $checkConv, : unable to evaluate scaled gradient
\end{verbatim}

\begin{verbatim}
## Warning in checkConv(attr(opt, "derivs"), opt$par, ctrl = control
## $checkConv, : Model failed to converge: degenerate Hessian with 2 negative
## eigenvalues
\end{verbatim}

\begin{verbatim}
## Warning in checkConv(attr(opt, "derivs"), opt$par, ctrl = control
## $checkConv, : unable to evaluate scaled gradient
\end{verbatim}

\begin{verbatim}
## Warning in checkConv(attr(opt, "derivs"), opt$par, ctrl = control
## $checkConv, : Model failed to converge: degenerate Hessian with 1 negative
## eigenvalues
\end{verbatim}

\begin{verbatim}
## Warning in checkConv(attr(opt, "derivs"), opt$par, ctrl = control
## $checkConv, : Model failed to converge with max|grad| = 0.00107783 (tol =
## 0.001, component 1)
\end{verbatim}

\begin{verbatim}
## Warning in checkConv(attr(opt, "derivs"), opt$par, ctrl = control
## $checkConv, : Model failed to converge with max|grad| = 0.0190203 (tol =
## 0.001, component 1)
\end{verbatim}

\begin{verbatim}
## Warning in checkConv(attr(opt, "derivs"), opt$par, ctrl = control
## $checkConv, : Model failed to converge with max|grad| = 0.00109175 (tol =
## 0.001, component 1)
\end{verbatim}

\begin{Shaded}
\begin{Highlighting}[]
\NormalTok{bootconf}
\end{Highlighting}
\end{Shaded}

\begin{verbatim}
##                      2.5 %    97.5 %
## .sig01        3.061080e-06 0.5698387
## (Intercept)  -8.525693e-01 0.2538298
## daycent      -2.189703e-02 0.1551001
## SEXM         -5.055166e-01 1.0011457
## daycent:SEXM -9.574260e-02 0.1378023
\end{verbatim}

\section{THIS IS ALL I'VE DONE SO
FAR\ldots{}\ldots{}}\label{this-is-all-ive-done-so-far}

\subsection{below is me exploring the bootstrap confidence intervals
with possible outliers removed from the datasets, but as noted above,
they don't seem to improve
anything.}\label{below-is-me-exploring-the-bootstrap-confidence-intervals-with-possible-outliers-removed-from-the-datasets-but-as-noted-above-they-dont-seem-to-improve-anything.}

\section{Graphing the data}\label{graphing-the-data}

We can plot the predictions of the model with the data as below. Note I
have added two types of predictions to the data frame, one, the
\texttt{fullpoppredict} is the population level predictions (average of
all the individuals, one trendline for each sex), while
\texttt{fullcondpredict} is the prediction at the level of each of the
individuals (one trendline for each individual).

\begin{Shaded}
\begin{Highlighting}[]
\KeywordTok{library}\NormalTok{(ggplot2)}
\KeywordTok{theme_set}\NormalTok{(}\KeywordTok{theme_bw}\NormalTok{())}
\NormalTok{exp1}\OperatorTok{$}\NormalTok{fullpoppredictS1<-}\StringTok{ }\KeywordTok{predict}\NormalTok{(SUCCESS1glmmfull, }
                           \DataTypeTok{re.form=}\OtherTok{NA}\NormalTok{,}\DataTypeTok{type=}\StringTok{"response"}\NormalTok{)}
\NormalTok{exp1}\OperatorTok{$}\NormalTok{fullcondpredictS1<-}\StringTok{ }\KeywordTok{predict}\NormalTok{(SUCCESS1glmmfull, }
                           \DataTypeTok{type=}\StringTok{"response"}\NormalTok{)}

\NormalTok{SUCC1 <-}\StringTok{ }\KeywordTok{ggplot}\NormalTok{(exp1,}\KeywordTok{aes}\NormalTok{(}\DataTypeTok{x=}\NormalTok{DAY,}\DataTypeTok{y=}\NormalTok{SUCCESS,}\DataTypeTok{color=}\NormalTok{SEX))}\OperatorTok{+}
\StringTok{  }\KeywordTok{geom_point}\NormalTok{(}\DataTypeTok{position=}\KeywordTok{position_jitter}\NormalTok{(}\DataTypeTok{height=}\NormalTok{.}\DecValTok{15}\NormalTok{,}\DataTypeTok{width=}\NormalTok{.}\DecValTok{15}\NormalTok{))}\OperatorTok{+}
\StringTok{  }\KeywordTok{geom_line}\NormalTok{(}\KeywordTok{aes}\NormalTok{(}\DataTypeTok{y=}\NormalTok{fullpoppredictS1}\OperatorTok{*}\DecValTok{3}\NormalTok{), }\DataTypeTok{size =} \DecValTok{2}\NormalTok{)}\OperatorTok{+}
\StringTok{  }\KeywordTok{geom_line}\NormalTok{(}\KeywordTok{aes}\NormalTok{(}\DataTypeTok{y=}\NormalTok{fullcondpredictS1}\OperatorTok{*}\DecValTok{3}\NormalTok{,}\DataTypeTok{group=}\NormalTok{SUB),}\DataTypeTok{alpha=}\NormalTok{.}\DecValTok{25}\NormalTok{)}\OperatorTok{+}\KeywordTok{scale_x_discrete}\NormalTok{(}\DataTypeTok{limit =} \KeywordTok{c}\NormalTok{(}\StringTok{"1"}\NormalTok{, }\StringTok{"2"}\NormalTok{, }\StringTok{"3"}\NormalTok{, }\StringTok{"4"}\NormalTok{, }\StringTok{"5"}\NormalTok{, }\StringTok{"6"}\NormalTok{, }\StringTok{"7"}\NormalTok{, }\StringTok{"8"}\NormalTok{, }\StringTok{"9"}\NormalTok{, }\StringTok{"10"}\NormalTok{))}
\NormalTok{SUCC1 }\OperatorTok{+}\StringTok{ }\KeywordTok{theme}\NormalTok{(}\DataTypeTok{panel.grid.minor =} \KeywordTok{element_blank}\NormalTok{(), }\DataTypeTok{panel.background =} \KeywordTok{element_rect}\NormalTok{(}\DataTypeTok{fill=}\StringTok{"gray95"}\NormalTok{), }\DataTypeTok{panel.grid.major =} \KeywordTok{element_line}\NormalTok{(}\DataTypeTok{color =} \StringTok{"white"}\NormalTok{), }\DataTypeTok{axis.text =} \KeywordTok{element_text}\NormalTok{(}\DataTypeTok{size =} \StringTok{"12"}\NormalTok{, }\DataTypeTok{face =} \StringTok{"bold"}\NormalTok{), }\DataTypeTok{axis.title =} \KeywordTok{element_text}\NormalTok{(}\DataTypeTok{size =} \StringTok{"22"}\NormalTok{, }\DataTypeTok{face =} \StringTok{"bold"}\NormalTok{), }\DataTypeTok{legend.title =} \KeywordTok{element_text}\NormalTok{(}\DataTypeTok{size =} \StringTok{"16"}\NormalTok{, }\DataTypeTok{face =} \StringTok{"bold"}\NormalTok{), }\DataTypeTok{legend.text =} \KeywordTok{element_text}\NormalTok{(}\DataTypeTok{size =} \StringTok{"12"}\NormalTok{, }\DataTypeTok{face =} \StringTok{"italic"}\NormalTok{))}
\end{Highlighting}
\end{Shaded}

\includegraphics{FROGDATA_files/figure-latex/unnamed-chunk-24-1.pdf}

\begin{Shaded}
\begin{Highlighting}[]
\KeywordTok{theme_set}\NormalTok{(}\KeywordTok{theme_bw}\NormalTok{())}
\NormalTok{exp2}\OperatorTok{$}\NormalTok{fullpoppredictS2 <-}\StringTok{ }\KeywordTok{predict}\NormalTok{(SUCCESS2glmmfull, }
                           \DataTypeTok{re.form=}\OtherTok{NA}\NormalTok{,}\DataTypeTok{type=}\StringTok{"response"}\NormalTok{)}
\NormalTok{exp2}\OperatorTok{$}\NormalTok{fullcondpredictS2 <-}\StringTok{ }\KeywordTok{predict}\NormalTok{(SUCCESS2glmmfull, }
                           \DataTypeTok{type=}\StringTok{"response"}\NormalTok{)}

\NormalTok{SUCC2 <-}\StringTok{ }\KeywordTok{ggplot}\NormalTok{(exp2,}\KeywordTok{aes}\NormalTok{(}\DataTypeTok{x=}\NormalTok{DAY,}\DataTypeTok{y=}\NormalTok{SUCCESS,}\DataTypeTok{color=}\NormalTok{SEX))}\OperatorTok{+}
\StringTok{  }\KeywordTok{geom_point}\NormalTok{(}\DataTypeTok{position=}\KeywordTok{position_jitter}\NormalTok{(}\DataTypeTok{height=}\NormalTok{.}\DecValTok{1}\NormalTok{,}\DataTypeTok{width=}\NormalTok{.}\DecValTok{1}\NormalTok{))}\OperatorTok{+}
\StringTok{  }\KeywordTok{geom_line}\NormalTok{(}\KeywordTok{aes}\NormalTok{(}\DataTypeTok{y=}\NormalTok{fullpoppredictS2}\OperatorTok{*}\DecValTok{3}\NormalTok{), }\DataTypeTok{size =} \DecValTok{2}\NormalTok{)}\OperatorTok{+}
\StringTok{  }\KeywordTok{geom_line}\NormalTok{(}\KeywordTok{aes}\NormalTok{(}\DataTypeTok{y=}\NormalTok{fullcondpredictS2}\OperatorTok{*}\DecValTok{3}\NormalTok{,}\DataTypeTok{group=}\NormalTok{SUB),}\DataTypeTok{alpha=}\NormalTok{.}\DecValTok{25}\NormalTok{)}\OperatorTok{+}\KeywordTok{scale_x_discrete}\NormalTok{(}\DataTypeTok{limit =} \KeywordTok{c}\NormalTok{(}\StringTok{"1"}\NormalTok{, }\StringTok{"2"}\NormalTok{, }\StringTok{"3"}\NormalTok{, }\StringTok{"4"}\NormalTok{, }\StringTok{"5"}\NormalTok{, }\StringTok{"6"}\NormalTok{, }\StringTok{"7"}\NormalTok{, }\StringTok{"8"}\NormalTok{, }\StringTok{"9"}\NormalTok{, }\StringTok{"10"}\NormalTok{))}
\NormalTok{SUCC2 }\OperatorTok{+}\StringTok{ }\KeywordTok{theme}\NormalTok{(}\DataTypeTok{panel.grid.minor =} \KeywordTok{element_blank}\NormalTok{(), }\DataTypeTok{panel.background =} \KeywordTok{element_rect}\NormalTok{(}\DataTypeTok{fill=}\StringTok{"gray95"}\NormalTok{), }\DataTypeTok{panel.grid.major =} \KeywordTok{element_line}\NormalTok{(}\DataTypeTok{color =} \StringTok{"white"}\NormalTok{), }\DataTypeTok{axis.text =} \KeywordTok{element_text}\NormalTok{(}\DataTypeTok{size =} \StringTok{"12"}\NormalTok{, }\DataTypeTok{face =} \StringTok{"bold"}\NormalTok{), }\DataTypeTok{axis.title =} \KeywordTok{element_text}\NormalTok{(}\DataTypeTok{size =} \StringTok{"22"}\NormalTok{, }\DataTypeTok{face =} \StringTok{"bold"}\NormalTok{), }\DataTypeTok{legend.title =} \KeywordTok{element_text}\NormalTok{(}\DataTypeTok{size =} \StringTok{"16"}\NormalTok{, }\DataTypeTok{face =} \StringTok{"bold"}\NormalTok{), }\DataTypeTok{legend.text =} \KeywordTok{element_text}\NormalTok{(}\DataTypeTok{size =} \StringTok{"12"}\NormalTok{, }\DataTypeTok{face =} \StringTok{"italic"}\NormalTok{))}
\end{Highlighting}
\end{Shaded}

\includegraphics{FROGDATA_files/figure-latex/unnamed-chunk-25-1.pdf}

\begin{Shaded}
\begin{Highlighting}[]
\KeywordTok{theme_set}\NormalTok{(}\KeywordTok{theme_bw}\NormalTok{())}
\NormalTok{exp1}\OperatorTok{$}\NormalTok{fullpoppredictNC1<-}\StringTok{ }\KeywordTok{predict}\NormalTok{(NC1glmmfull, }
                           \DataTypeTok{re.form=}\OtherTok{NA}\NormalTok{,}\DataTypeTok{type=}\StringTok{"response"}\NormalTok{)}
\NormalTok{exp1}\OperatorTok{$}\NormalTok{fullcondpredictNC1<-}\StringTok{ }\KeywordTok{predict}\NormalTok{(NC1glmmfull, }
                           \DataTypeTok{type=}\StringTok{"response"}\NormalTok{)}

\NormalTok{NC1 <-}\StringTok{ }\KeywordTok{ggplot}\NormalTok{(exp1,}\KeywordTok{aes}\NormalTok{(}\DataTypeTok{x=}\NormalTok{DAY,}\DataTypeTok{y=}\NormalTok{NC_errors,}\DataTypeTok{color=}\NormalTok{SEX))}\OperatorTok{+}
\StringTok{  }\KeywordTok{geom_point}\NormalTok{(}\DataTypeTok{position=}\KeywordTok{position_jitter}\NormalTok{(}\DataTypeTok{height=}\NormalTok{.}\DecValTok{15}\NormalTok{,}\DataTypeTok{width=}\NormalTok{.}\DecValTok{15}\NormalTok{))}\OperatorTok{+}
\StringTok{  }\KeywordTok{geom_line}\NormalTok{(}\KeywordTok{aes}\NormalTok{(}\DataTypeTok{y=}\NormalTok{fullpoppredictNC1}\OperatorTok{*}\DecValTok{3}\NormalTok{), }\DataTypeTok{size =} \DecValTok{2}\NormalTok{)}\OperatorTok{+}
\StringTok{  }\KeywordTok{geom_line}\NormalTok{(}\KeywordTok{aes}\NormalTok{(}\DataTypeTok{y=}\NormalTok{fullcondpredictNC1}\OperatorTok{*}\DecValTok{3}\NormalTok{,}\DataTypeTok{group=}\NormalTok{SUB),}\DataTypeTok{alpha=}\NormalTok{.}\DecValTok{25}\NormalTok{)}\OperatorTok{+}\KeywordTok{scale_x_discrete}\NormalTok{(}\DataTypeTok{limit =} \KeywordTok{c}\NormalTok{(}\StringTok{"1"}\NormalTok{, }\StringTok{"2"}\NormalTok{, }\StringTok{"3"}\NormalTok{, }\StringTok{"4"}\NormalTok{, }\StringTok{"5"}\NormalTok{, }\StringTok{"6"}\NormalTok{, }\StringTok{"7"}\NormalTok{, }\StringTok{"8"}\NormalTok{, }\StringTok{"9"}\NormalTok{, }\StringTok{"10"}\NormalTok{))}
\NormalTok{NC1 }\OperatorTok{+}\StringTok{ }\KeywordTok{theme}\NormalTok{(}\DataTypeTok{panel.grid.minor =} \KeywordTok{element_blank}\NormalTok{(), }\DataTypeTok{panel.background =} \KeywordTok{element_rect}\NormalTok{(}\DataTypeTok{fill=}\StringTok{"gray95"}\NormalTok{), }\DataTypeTok{panel.grid.major =} \KeywordTok{element_line}\NormalTok{(}\DataTypeTok{color =} \StringTok{"white"}\NormalTok{), }\DataTypeTok{axis.text =} \KeywordTok{element_text}\NormalTok{(}\DataTypeTok{size =} \StringTok{"12"}\NormalTok{, }\DataTypeTok{face =} \StringTok{"bold"}\NormalTok{), }\DataTypeTok{axis.title =} \KeywordTok{element_text}\NormalTok{(}\DataTypeTok{size =} \StringTok{"22"}\NormalTok{, }\DataTypeTok{face =} \StringTok{"bold"}\NormalTok{), }\DataTypeTok{legend.title =} \KeywordTok{element_text}\NormalTok{(}\DataTypeTok{size =} \StringTok{"16"}\NormalTok{, }\DataTypeTok{face =} \StringTok{"bold"}\NormalTok{), }\DataTypeTok{legend.text =} \KeywordTok{element_text}\NormalTok{(}\DataTypeTok{size =} \StringTok{"12"}\NormalTok{, }\DataTypeTok{face =} \StringTok{"italic"}\NormalTok{))}
\end{Highlighting}
\end{Shaded}

\includegraphics{FROGDATA_files/figure-latex/unnamed-chunk-26-1.pdf}

\begin{Shaded}
\begin{Highlighting}[]
\KeywordTok{theme_set}\NormalTok{(}\KeywordTok{theme_bw}\NormalTok{())}
\NormalTok{exp2}\OperatorTok{$}\NormalTok{fullpoppredictNC2<-}\StringTok{ }\KeywordTok{predict}\NormalTok{(NC2glmmfull, }
                           \DataTypeTok{re.form=}\OtherTok{NA}\NormalTok{,}\DataTypeTok{type=}\StringTok{"response"}\NormalTok{)}
\NormalTok{exp2}\OperatorTok{$}\NormalTok{fullcondpredictNC2<-}\StringTok{ }\KeywordTok{predict}\NormalTok{(NC2glmmfull, }
                           \DataTypeTok{type=}\StringTok{"response"}\NormalTok{)}

\NormalTok{NC2 <-}\StringTok{ }\KeywordTok{ggplot}\NormalTok{(exp2,}\KeywordTok{aes}\NormalTok{(}\DataTypeTok{x=}\NormalTok{DAY,}\DataTypeTok{y=}\NormalTok{NC_errors,}\DataTypeTok{color=}\NormalTok{SEX))}\OperatorTok{+}
\StringTok{  }\KeywordTok{geom_point}\NormalTok{(}\DataTypeTok{position=}\KeywordTok{position_jitter}\NormalTok{(}\DataTypeTok{height=}\NormalTok{.}\DecValTok{15}\NormalTok{,}\DataTypeTok{width=}\NormalTok{.}\DecValTok{15}\NormalTok{))}\OperatorTok{+}
\StringTok{  }\KeywordTok{geom_line}\NormalTok{(}\KeywordTok{aes}\NormalTok{(}\DataTypeTok{y=}\NormalTok{fullpoppredictNC2}\OperatorTok{*}\DecValTok{3}\NormalTok{), }\DataTypeTok{size =} \DecValTok{2}\NormalTok{)}\OperatorTok{+}
\StringTok{  }\KeywordTok{geom_line}\NormalTok{(}\KeywordTok{aes}\NormalTok{(}\DataTypeTok{y=}\NormalTok{fullcondpredictNC2}\OperatorTok{*}\DecValTok{3}\NormalTok{,}\DataTypeTok{group=}\NormalTok{SUB),}\DataTypeTok{alpha=}\NormalTok{.}\DecValTok{25}\NormalTok{)}\OperatorTok{+}\KeywordTok{scale_x_discrete}\NormalTok{(}\DataTypeTok{limit =} \KeywordTok{c}\NormalTok{(}\StringTok{"1"}\NormalTok{, }\StringTok{"2"}\NormalTok{, }\StringTok{"3"}\NormalTok{, }\StringTok{"4"}\NormalTok{, }\StringTok{"5"}\NormalTok{, }\StringTok{"6"}\NormalTok{, }\StringTok{"7"}\NormalTok{, }\StringTok{"8"}\NormalTok{, }\StringTok{"9"}\NormalTok{, }\StringTok{"10"}\NormalTok{))}
\NormalTok{NC2 }\OperatorTok{+}\StringTok{ }\KeywordTok{theme}\NormalTok{(}\DataTypeTok{panel.grid.minor =} \KeywordTok{element_blank}\NormalTok{(), }\DataTypeTok{panel.background =} \KeywordTok{element_rect}\NormalTok{(}\DataTypeTok{fill=}\StringTok{"gray95"}\NormalTok{), }\DataTypeTok{panel.grid.major =} \KeywordTok{element_line}\NormalTok{(}\DataTypeTok{color =} \StringTok{"white"}\NormalTok{), }\DataTypeTok{axis.text =} \KeywordTok{element_text}\NormalTok{(}\DataTypeTok{size =} \StringTok{"12"}\NormalTok{, }\DataTypeTok{face =} \StringTok{"bold"}\NormalTok{), }\DataTypeTok{axis.title =} \KeywordTok{element_text}\NormalTok{(}\DataTypeTok{size =} \StringTok{"22"}\NormalTok{, }\DataTypeTok{face =} \StringTok{"bold"}\NormalTok{), }\DataTypeTok{legend.title =} \KeywordTok{element_text}\NormalTok{(}\DataTypeTok{size =} \StringTok{"16"}\NormalTok{, }\DataTypeTok{face =} \StringTok{"bold"}\NormalTok{), }\DataTypeTok{legend.text =} \KeywordTok{element_text}\NormalTok{(}\DataTypeTok{size =} \StringTok{"12"}\NormalTok{, }\DataTypeTok{face =} \StringTok{"italic"}\NormalTok{))}
\end{Highlighting}
\end{Shaded}

\includegraphics{FROGDATA_files/figure-latex/unnamed-chunk-27-1.pdf}

\begin{Shaded}
\begin{Highlighting}[]
\KeywordTok{theme_set}\NormalTok{(}\KeywordTok{theme_bw}\NormalTok{())}
\NormalTok{exp1}\OperatorTok{$}\NormalTok{fullpoppredictP1<-}\StringTok{ }\KeywordTok{predict}\NormalTok{(P1glmmfull, }
                           \DataTypeTok{re.form=}\OtherTok{NA}\NormalTok{,}\DataTypeTok{type=}\StringTok{"response"}\NormalTok{)}
\NormalTok{exp1}\OperatorTok{$}\NormalTok{fullcondpredictP1<-}\StringTok{ }\KeywordTok{predict}\NormalTok{(P1glmmfull, }
                           \DataTypeTok{type=}\StringTok{"response"}\NormalTok{)}

\NormalTok{P1 <-}\StringTok{ }\KeywordTok{ggplot}\NormalTok{(exp1,}\KeywordTok{aes}\NormalTok{(}\DataTypeTok{x=}\NormalTok{DAY,}\DataTypeTok{y=}\NormalTok{P_errors,}\DataTypeTok{color=}\NormalTok{SEX))}\OperatorTok{+}
\StringTok{  }\KeywordTok{geom_point}\NormalTok{(}\DataTypeTok{position=}\KeywordTok{position_jitter}\NormalTok{(}\DataTypeTok{height=}\NormalTok{.}\DecValTok{15}\NormalTok{,}\DataTypeTok{width=}\NormalTok{.}\DecValTok{15}\NormalTok{))}\OperatorTok{+}
\StringTok{  }\KeywordTok{geom_line}\NormalTok{(}\KeywordTok{aes}\NormalTok{(}\DataTypeTok{y=}\NormalTok{fullpoppredictP1), }\DataTypeTok{size =} \DecValTok{2}\NormalTok{)}\OperatorTok{+}
\StringTok{  }\KeywordTok{geom_line}\NormalTok{(}\KeywordTok{aes}\NormalTok{(}\DataTypeTok{y=}\NormalTok{fullcondpredictP1,}\DataTypeTok{group=}\NormalTok{SUB),}\DataTypeTok{alpha=}\NormalTok{.}\DecValTok{25}\NormalTok{)}\OperatorTok{+}\KeywordTok{scale_x_discrete}\NormalTok{(}\DataTypeTok{limit =} \KeywordTok{c}\NormalTok{(}\StringTok{"1"}\NormalTok{, }\StringTok{"2"}\NormalTok{, }\StringTok{"3"}\NormalTok{, }\StringTok{"4"}\NormalTok{, }\StringTok{"5"}\NormalTok{, }\StringTok{"6"}\NormalTok{, }\StringTok{"7"}\NormalTok{, }\StringTok{"8"}\NormalTok{, }\StringTok{"9"}\NormalTok{, }\StringTok{"10"}\NormalTok{))}
\NormalTok{P1 }\OperatorTok{+}\StringTok{ }\KeywordTok{theme}\NormalTok{(}\DataTypeTok{panel.grid.minor =} \KeywordTok{element_blank}\NormalTok{(), }\DataTypeTok{panel.background =} \KeywordTok{element_rect}\NormalTok{(}\DataTypeTok{fill=}\StringTok{"gray95"}\NormalTok{), }\DataTypeTok{panel.grid.major =} \KeywordTok{element_line}\NormalTok{(}\DataTypeTok{color =} \StringTok{"white"}\NormalTok{), }\DataTypeTok{axis.text =} \KeywordTok{element_text}\NormalTok{(}\DataTypeTok{size =} \StringTok{"12"}\NormalTok{, }\DataTypeTok{face =} \StringTok{"bold"}\NormalTok{), }\DataTypeTok{axis.title =} \KeywordTok{element_text}\NormalTok{(}\DataTypeTok{size =} \StringTok{"22"}\NormalTok{, }\DataTypeTok{face =} \StringTok{"bold"}\NormalTok{), }\DataTypeTok{legend.title =} \KeywordTok{element_text}\NormalTok{(}\DataTypeTok{size =} \StringTok{"16"}\NormalTok{, }\DataTypeTok{face =} \StringTok{"bold"}\NormalTok{), }\DataTypeTok{legend.text =} \KeywordTok{element_text}\NormalTok{(}\DataTypeTok{size =} \StringTok{"12"}\NormalTok{, }\DataTypeTok{face =} \StringTok{"italic"}\NormalTok{))}
\end{Highlighting}
\end{Shaded}

\includegraphics{FROGDATA_files/figure-latex/unnamed-chunk-28-1.pdf}

\begin{Shaded}
\begin{Highlighting}[]
\KeywordTok{theme_set}\NormalTok{(}\KeywordTok{theme_bw}\NormalTok{())}
\NormalTok{exp2}\OperatorTok{$}\NormalTok{fullpoppredictP2<-}\StringTok{ }\KeywordTok{predict}\NormalTok{(P2glmmfull, }
                           \DataTypeTok{re.form=}\OtherTok{NA}\NormalTok{,}\DataTypeTok{type=}\StringTok{"response"}\NormalTok{)}
\NormalTok{exp2}\OperatorTok{$}\NormalTok{fullcondpredictP2<-}\StringTok{ }\KeywordTok{predict}\NormalTok{(P2glmmfull, }
                           \DataTypeTok{type=}\StringTok{"response"}\NormalTok{)}

\NormalTok{P2 <-}\StringTok{ }\KeywordTok{ggplot}\NormalTok{(exp2,}\KeywordTok{aes}\NormalTok{(}\DataTypeTok{x=}\NormalTok{DAY,}\DataTypeTok{y=}\NormalTok{P_errors,}\DataTypeTok{color=}\NormalTok{SEX))}\OperatorTok{+}
\StringTok{  }\KeywordTok{geom_point}\NormalTok{(}\DataTypeTok{position=}\KeywordTok{position_jitter}\NormalTok{(}\DataTypeTok{height=}\NormalTok{.}\DecValTok{15}\NormalTok{,}\DataTypeTok{width=}\NormalTok{.}\DecValTok{15}\NormalTok{))}\OperatorTok{+}
\StringTok{  }\KeywordTok{geom_line}\NormalTok{(}\KeywordTok{aes}\NormalTok{(}\DataTypeTok{y=}\NormalTok{fullpoppredictP2), }\DataTypeTok{size =} \DecValTok{2}\NormalTok{)}\OperatorTok{+}
\StringTok{  }\KeywordTok{geom_line}\NormalTok{(}\KeywordTok{aes}\NormalTok{(}\DataTypeTok{y=}\NormalTok{fullcondpredictP2,}\DataTypeTok{group=}\NormalTok{SUB),}\DataTypeTok{alpha=}\NormalTok{.}\DecValTok{25}\NormalTok{)}\OperatorTok{+}\KeywordTok{scale_x_discrete}\NormalTok{(}\DataTypeTok{limit =} \KeywordTok{c}\NormalTok{(}\StringTok{"1"}\NormalTok{, }\StringTok{"2"}\NormalTok{, }\StringTok{"3"}\NormalTok{, }\StringTok{"4"}\NormalTok{, }\StringTok{"5"}\NormalTok{, }\StringTok{"6"}\NormalTok{, }\StringTok{"7"}\NormalTok{, }\StringTok{"8"}\NormalTok{, }\StringTok{"9"}\NormalTok{, }\StringTok{"10"}\NormalTok{))}
\NormalTok{P2 }\OperatorTok{+}\StringTok{ }\KeywordTok{theme}\NormalTok{(}\DataTypeTok{panel.grid.minor =} \KeywordTok{element_blank}\NormalTok{(), }\DataTypeTok{panel.background =} \KeywordTok{element_rect}\NormalTok{(}\DataTypeTok{fill=}\StringTok{"gray95"}\NormalTok{), }\DataTypeTok{panel.grid.major =} \KeywordTok{element_line}\NormalTok{(}\DataTypeTok{color =} \StringTok{"white"}\NormalTok{), }\DataTypeTok{axis.text =} \KeywordTok{element_text}\NormalTok{(}\DataTypeTok{size =} \StringTok{"12"}\NormalTok{, }\DataTypeTok{face =} \StringTok{"bold"}\NormalTok{), }\DataTypeTok{axis.title =} \KeywordTok{element_text}\NormalTok{(}\DataTypeTok{size =} \StringTok{"22"}\NormalTok{, }\DataTypeTok{face =} \StringTok{"bold"}\NormalTok{), }\DataTypeTok{legend.title =} \KeywordTok{element_text}\NormalTok{(}\DataTypeTok{size =} \StringTok{"16"}\NormalTok{, }\DataTypeTok{face =} \StringTok{"bold"}\NormalTok{), }\DataTypeTok{legend.text =} \KeywordTok{element_text}\NormalTok{(}\DataTypeTok{size =} \StringTok{"12"}\NormalTok{, }\DataTypeTok{face =} \StringTok{"italic"}\NormalTok{))}
\end{Highlighting}
\end{Shaded}

\includegraphics{FROGDATA_files/figure-latex/unnamed-chunk-29-1.pdf}

\begin{Shaded}
\begin{Highlighting}[]
\NormalTok{exp1means <-}\StringTok{ }\KeywordTok{read.csv}\NormalTok{(}\StringTok{"exp1means.csv"}\NormalTok{, }\DataTypeTok{header =} \OtherTok{TRUE}\NormalTok{)}

\NormalTok{MEANS <-}\StringTok{ }\KeywordTok{ggplot}\NormalTok{(exp1means,}\KeywordTok{aes}\NormalTok{(}\DataTypeTok{x=}\NormalTok{DAY, }\DataTypeTok{y=}\NormalTok{MEAN, }\DataTypeTok{group=}\NormalTok{SEX, }\DataTypeTok{color=}\NormalTok{SEX))}\OperatorTok{+}
\StringTok{  }\KeywordTok{geom_point}\NormalTok{()}\OperatorTok{+}\KeywordTok{geom_errorbar}\NormalTok{(}\KeywordTok{aes}\NormalTok{(}\DataTypeTok{ymin=}\NormalTok{MEAN}\OperatorTok{-}\NormalTok{SD, }\DataTypeTok{ymax=}\NormalTok{MEAN}\OperatorTok{+}\NormalTok{SD), }\DataTypeTok{width=}\FloatTok{0.2}\NormalTok{, }\DataTypeTok{position=}\KeywordTok{position_dodge}\NormalTok{(}\FloatTok{0.05}\NormalTok{)) }\OperatorTok{+}\KeywordTok{geom_line}\NormalTok{(}\KeywordTok{aes}\NormalTok{(}\DataTypeTok{y=}\NormalTok{MEAN), }\DataTypeTok{size =} \FloatTok{1.5}\NormalTok{) }\OperatorTok{+}\StringTok{ }\KeywordTok{scale_x_discrete}\NormalTok{(}\DataTypeTok{limit =} \KeywordTok{c}\NormalTok{(}\StringTok{"1"}\NormalTok{, }\StringTok{"2"}\NormalTok{, }\StringTok{"3"}\NormalTok{, }\StringTok{"4"}\NormalTok{, }\StringTok{"5"}\NormalTok{, }\StringTok{"6"}\NormalTok{, }\StringTok{"7"}\NormalTok{, }\StringTok{"8"}\NormalTok{, }\StringTok{"9"}\NormalTok{, }\StringTok{"10"}\NormalTok{))}
\NormalTok{MEANS }\OperatorTok{+}\StringTok{ }\KeywordTok{scale_y_continuous}\NormalTok{(}\DataTypeTok{breaks =} \KeywordTok{seq}\NormalTok{(}\DecValTok{0}\NormalTok{,}\DecValTok{1}\NormalTok{,}\FloatTok{0.1}\NormalTok{)) }\OperatorTok{+}\StringTok{ }\KeywordTok{theme}\NormalTok{(}\DataTypeTok{panel.grid.minor =} \KeywordTok{element_blank}\NormalTok{(), }\DataTypeTok{panel.background =} \KeywordTok{element_rect}\NormalTok{(}\DataTypeTok{fill=}\StringTok{"gray95"}\NormalTok{), }\DataTypeTok{panel.grid.major =} \KeywordTok{element_line}\NormalTok{(}\DataTypeTok{color =} \StringTok{"white"}\NormalTok{), }\DataTypeTok{axis.text =} \KeywordTok{element_text}\NormalTok{(}\DataTypeTok{size =} \StringTok{"12"}\NormalTok{, }\DataTypeTok{face =} \StringTok{"bold"}\NormalTok{), }\DataTypeTok{axis.title =} \KeywordTok{element_text}\NormalTok{(}\DataTypeTok{size =} \StringTok{"22"}\NormalTok{, }\DataTypeTok{face =} \StringTok{"bold"}\NormalTok{), }\DataTypeTok{legend.title =} \KeywordTok{element_text}\NormalTok{(}\DataTypeTok{size =} \StringTok{"16"}\NormalTok{, }\DataTypeTok{face =} \StringTok{"bold"}\NormalTok{), }\DataTypeTok{legend.text =} \KeywordTok{element_text}\NormalTok{(}\DataTypeTok{size =} \StringTok{"12"}\NormalTok{, }\DataTypeTok{face =} \StringTok{"italic"}\NormalTok{))}
\end{Highlighting}
\end{Shaded}

\includegraphics{FROGDATA_files/figure-latex/unnamed-chunk-30-1.pdf}

\begin{Shaded}
\begin{Highlighting}[]
\NormalTok{exp2means <-}\StringTok{ }\KeywordTok{read.csv}\NormalTok{(}\StringTok{"exp2means.csv"}\NormalTok{, }\DataTypeTok{header =} \OtherTok{TRUE}\NormalTok{)}

\NormalTok{MEANS <-}\StringTok{ }\KeywordTok{ggplot}\NormalTok{(exp2means,}\KeywordTok{aes}\NormalTok{(}\DataTypeTok{x=}\NormalTok{DAY, }\DataTypeTok{y=}\NormalTok{MEAN, }\DataTypeTok{group=}\NormalTok{SEX, }\DataTypeTok{color=}\NormalTok{SEX))}\OperatorTok{+}
\StringTok{  }\KeywordTok{geom_point}\NormalTok{()}\OperatorTok{+}\KeywordTok{geom_errorbar}\NormalTok{(}\KeywordTok{aes}\NormalTok{(}\DataTypeTok{ymin=}\NormalTok{MEAN}\OperatorTok{-}\NormalTok{SD, }\DataTypeTok{ymax=}\NormalTok{MEAN}\OperatorTok{+}\NormalTok{SD), }\DataTypeTok{width=}\FloatTok{0.2}\NormalTok{, }\DataTypeTok{position=}\KeywordTok{position_dodge}\NormalTok{(}\FloatTok{0.05}\NormalTok{)) }\OperatorTok{+}\KeywordTok{geom_line}\NormalTok{(}\KeywordTok{aes}\NormalTok{(}\DataTypeTok{y=}\NormalTok{MEAN), }\DataTypeTok{size =} \FloatTok{1.5}\NormalTok{) }\OperatorTok{+}\StringTok{ }\KeywordTok{scale_x_discrete}\NormalTok{(}\DataTypeTok{limit =} \KeywordTok{c}\NormalTok{(}\StringTok{"1"}\NormalTok{, }\StringTok{"2"}\NormalTok{, }\StringTok{"3"}\NormalTok{, }\StringTok{"4"}\NormalTok{, }\StringTok{"5"}\NormalTok{, }\StringTok{"6"}\NormalTok{, }\StringTok{"7"}\NormalTok{, }\StringTok{"8"}\NormalTok{, }\StringTok{"9"}\NormalTok{, }\StringTok{"10"}\NormalTok{))}
\NormalTok{MEANS }\OperatorTok{+}\StringTok{ }\KeywordTok{scale_y_continuous}\NormalTok{(}\DataTypeTok{breaks =} \KeywordTok{seq}\NormalTok{(}\DecValTok{0}\NormalTok{,}\DecValTok{1}\NormalTok{,}\FloatTok{0.1}\NormalTok{)) }\OperatorTok{+}\StringTok{ }\KeywordTok{theme}\NormalTok{(}\DataTypeTok{panel.grid.minor =} \KeywordTok{element_blank}\NormalTok{(), }\DataTypeTok{panel.background =} \KeywordTok{element_rect}\NormalTok{(}\DataTypeTok{fill=}\StringTok{"gray95"}\NormalTok{), }\DataTypeTok{panel.grid.major =} \KeywordTok{element_line}\NormalTok{(}\DataTypeTok{color =} \StringTok{"white"}\NormalTok{), }\DataTypeTok{axis.text =} \KeywordTok{element_text}\NormalTok{(}\DataTypeTok{size =} \StringTok{"12"}\NormalTok{, }\DataTypeTok{face =} \StringTok{"bold"}\NormalTok{), }\DataTypeTok{axis.title =} \KeywordTok{element_text}\NormalTok{(}\DataTypeTok{size =} \StringTok{"22"}\NormalTok{, }\DataTypeTok{face =} \StringTok{"bold"}\NormalTok{), }\DataTypeTok{legend.title =} \KeywordTok{element_text}\NormalTok{(}\DataTypeTok{size =} \StringTok{"16"}\NormalTok{, }\DataTypeTok{face =} \StringTok{"bold"}\NormalTok{), }\DataTypeTok{legend.text =} \KeywordTok{element_text}\NormalTok{(}\DataTypeTok{size =} \StringTok{"12"}\NormalTok{, }\DataTypeTok{face =} \StringTok{"italic"}\NormalTok{))}
\end{Highlighting}
\end{Shaded}

\includegraphics{FROGDATA_files/figure-latex/unnamed-chunk-31-1.pdf}


\end{document}
